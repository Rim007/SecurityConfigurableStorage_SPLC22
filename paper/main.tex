% !TeX spellcheck = en_US
\PassOptionsToPackage{table}{xcolor}

\documentclass[sigconf,review,anonymous]{acmart}

\usepackage[utf8]{inputenc}
\usepackage[T1]{fontenc}

\usepackage{combelow}
\usepackage{tabularx}
\newcolumntype{L}[1]{>{\raggedright\arraybackslash}p{#1}}
\newcolumntype{C}[1]{>{\centering\arraybackslash}p{#1}}
\newcolumntype{R}[1]{>{\raggedleft\arraybackslash}p{#1}}
\newlength{\fixedColWidth}
\setlength{\fixedColWidth}{1.5em}
\usepackage{multicol}
\usepackage{multirow}
\usepackage{natbib}
\usepackage{hyperref}
\renewcommand{\sectionautorefname}{Section}
\renewcommand{\subsectionautorefname}{Section}
\usepackage{enumitem}
\usepackage{listings}
\usepackage{balance}
\usepackage{graphicx}
\usepackage{multirow}
\usepackage{relsize,xspace}
\usepackage{booktabs}
\usepackage{adjustbox}
\usepackage[super]{nth}
\usepackage{csquotes}
\usepackage{acronym}
\usepackage{xspace}
\usepackage{pifont}
\usepackage{harveyballs}

\newcommand{\searchString}[1]{\noindent \begin{center} \vspace{-.2em} \hspace*{0.5ex}\colorbox{lightgray}{\parbox{0.95\columnwidth}{{\texttt{#1}}}} \vspace{-.5em} \hspace*{0.5ex} \end{center}}

\newcommand\tabrotate[1]{\rotatebox{90}{#1\hspace{\tabcolsep}}}

% REMOVE FOR FINAL %%
\usepackage[sharp]{easylist}
\AtBeginEnvironment{easylist}{\ListProperties(	Start1=1,%
	Hang=true,Space=2pt,Space*=2pt,Hide=1000,%
	Align=fixed,
	Margin1=0.0em,Style1*=\textbullet\hskip .5em,%
	Margin2=1.0em,Style2*=--\hskip .5em,%
	Margin3=2.0em,Style3*=$\ast$\hskip .5em,%
	Margin4=3.0em,Style4*=$\cdot$\hskip .5em,%
	Margin5=4.0em,Style5*=$\cdot$\hskip .5em,%
	Margin6=5.0em,Style6*=$\cdot$\hskip .5em,%
	Margin7=6.0em,Style7*=$\cdot$\hskip .5em)
}%

\acmConference[SPLC’22]{26th ACM International Systems and Software Product Lines Conference}{12-16 September, 2022}{Graz, Austria}

\newcommand{\myPar}[1]{\noindent\textbf{#1.}}
\newcommand{\todo}[1]{\noindent\textcolor{red}{\small\textbf{Todo: #1}}}
\newcommand{\remark}[1]{\noindent\textcolor{blue}{\small\textbf{Remark: #1}}}

\begin{document}

%\title[A Systematic Mapping Study on Security of Configurable Systems and Their Data Storage]{A Systematic Mapping Study on Security\\ of Configurable Systems and Their Data Storage}
%
\title{A Systematic Mapping Study of Security Concepts for Configurable Data Storages}
	
\author{Richard May}
\orcid{0000-0001-7186-404X}
\affiliation{%
	\institution{Harz University Wernigerode, Germany}
	\city{}
	\country{}}
\email{rmay@hs-harz.de}	

\author{Christian Biermann}
\affiliation{%
	\institution{Harz University Wernigerode, Germany}
	\city{}
	\country{}}
\email{cbiermann@hs-harz.de}
	
\author{Jacob Krüger}
\orcid{0000-0002-0283-248X}
\affiliation{%
	\institution{Ruhr-University Bochum, Germany}
	\city{}
	\country{}}
\email{jacob.krueger@rub.de}

\author{Gunter Saake}
\affiliation{%
	\institution{Otto-von-Guericke University Magdeburg, Germany}
	\city{}
	\country{}}
\email{saake@ovgu.de}

\author{Thomas Leich}
\affiliation{
	\institution{Harz University Wernigerode, Germany}
	\city{}
	\country{}
}
\email{tleich@hs-harz.de}

\renewcommand{\shortauthors}{May et al.}

\begin{abstract}
Most modern software systems can be configured to fulfill specific customer requirements, adapting their behavior as required.
However, such adaptations also increase the need to consider security concerns, for instance, to avoid that unintended feature interactions cause a vulnerability that an attacker can exploit.
A particularly interesting aspect in this context are data storages (e.g., databases) used within the system, since the adapted behavior may change how (critical) data is collected, stored, processed, and accessed.
Unfortunately, there is no comprehensive overview of the state-of-the-art on security concerns of configurable data storages.
To address this gap, we conducted a systematic mapping study in which we analyzed 50 publications from the last decade (2013--2022).  
We compare these publications based on the configurable systems, data storages, and security concerns involved; using established classification criteria of the respective research fields.
Overall, we identified 16 open challenges, which we discuss in detail to guide future research.
Our key insight is that the security of configurable data storages seems to be under-explored and is rarely considered in a practice-oriented way, for instance, regarding relevant security standards. 
Furthermore, data storages and their security concerns are usually only mentioned briefly, even though they are either highly configurable or store critical data.
Our mapping study aims to help both practitioners and researchers to understand the current state-of-the-art research, identify open issues, and thus guide future research.
\end{abstract}


\begin{CCSXML}
	<ccs2012>
	<concept>
	<concept_id>10002978.10003006.10011634</concept_id>
	<concept_desc>Security and privacy~Vulnerability management</concept_desc>
	<concept_significance>500</concept_significance>
	</concept>
	<concept>
	<concept_id>10011007.10011074.10011092.10011096.10011097</concept_id>
	<concept_desc>Software and its engineering~Software product lines</concept_desc>
	<concept_significance>500</concept_significance>
	</concept>
	</ccs2012>
\end{CCSXML}

\ccsdesc[500]{Security and privacy~Vulnerability management}
\ccsdesc[500]{Software and its engineering~Software product lines}

\keywords{Security, Data Storage, Configurable Systems, Software Product Line Engineering, Mapping Study}

\maketitle

\section{Introduction} \label{sec:introduction}

In recent years, the amount of data (e.g., images, audios, logs) massively increases due to the growing complexity, interconnection, and interaction of software systems~\citep{sagiroglu2013big,oussous2018big}.
To handle this data in an appropriate way, it is stored, managed, or even processed in a variety of different data storage systems, ranging from centralized databases to on-demand software-as-a-service (SaaS) cloud storage to decentralized blockchains~\citep{gabel2017secure,cusumano2010cloud,shafagh2017towards}.
A growing amount of these systems, but also their underlying software systems, is configurable, i.e., these systems are based on variability allowing their adaptation to specific requirements, e.g., customer demands, industry standards, hardware limitations, or legal regulations.
Moreover, the configurability allows the derivation of families of similar but customized variants by enabling or disabling certain features.
So, since each variant of a system includes individual functions and potential interactions between them, it contributes to the uniqueness but also the complexity of every solution~\citep{pohl2005software}. 

However, the data to be stored is usually private or security sensitive.
Consequently, to protect this data, it is essential to process and store it in a secure way by relying on concrete security models or standards (e.g.,~\citet{ISOIEC270002018})~\citep{Sartakov2018STANliteLevel,Popa2011CryptDB:Processing}. 
Due to the uniqueness and complexity of configurable storage systems as well as their underlying configurable software systems based on a variety of variants, there is a particularly high risk for both configurable systems of becoming the subject of cyber attacks~\citep{Kenner2021MappingStudy}.
These attacks are often based on the exploitation of certain system vulnerabilities or malicious administrators, leading to unauthorized data access, data loss or even data manipulation~\cite{Gunawi2014WhatCloud, Campanile2017CloudifyingDomain}.

Consequently, research on configurable systems, particularly those based on software product lines (SPL), and security has been conducted in recent years.
This research includes both approaches, i.e., security and SPL, for configurable software systems and configurable storage systems.
For instance 

\todo{so, just for me to be clear, the goal is to investigate to what extent security concerns are considered/represented/... within the configuration space of data storage systems? I think we should be a bit clearer on the definitions. Also, I would change the title to not be too redundant with the previous paper ;) --- but I think I recall you said its a working title only}

\todo{Beispiele}


Although there exist extensive research on security and configurable systems, especially involving storage systems, we are missing an overview of current state-of-the-art research.
Consequently, there is no comprehensive systematization of such storage systems, their security and variability concerns, or future research directions.
By conducting a systematic mapping study, we aim to addressed this research gap~\cite{Kitchenham2015EvidenceSLR}.
Precisely, we analyzed 53 papers which are concerned with security, and SPL-related storage systems published on \textsc{Scopus}\footnote{www.scopus.com}, \textsc{IEEE Xplore}\footnote{ieeexplore.ieee.org}, and the \textsc{ACM Guide to Computing Literature}\footnote{dl.acm.org/search/advanced} within the last decade (2013--2022).
This way, we are able to summarize the current state-of-art research and emphasize research gaps as well as relevant research direction regarding the intersection of security and SPL-related storage systems.

Overall, we contribute the following with our study:
\begin{itemize}[nosep]
	\item A systematic and comprehensive overview of recent research on security in the context of SPL-related storage systems.
	
	\item A discussion on what properties are covered and for which properties further investigations is required.
	
	\item An open-access repository including the detailed study results to increase the study's overall comprehensibility and to ensure a replicability of our study.\footnote{\url{LINK}\label{ft:repo}}.
\end{itemize}


We argue that our study results and discussion findings can help both researchers and practitioners in understanding the recent state-of-the-art as well as open research directions.

\section{Background} \label{sec:background}

In the following section, we provide relevant background information on \emph{configurable systems}, \emph{data storages}, and \emph{security}.


\subsection{Configurable Systems}\label{sec:background:configsystems}

A configurable system is characterized by a number of features (i.e., user-visible functionalities~\cite{Apel2013FOSPL}) that can be configured to meet customer needs, such as user requirements, hardware constraints, or legal regulations.
We classify configurable systems into two categories: configurable software systems (e.g., plugin-based systems) and configurable storage systems (e.g., cloud-based systems).
Usually, such systems rely on concepts, methods, and techniques related to SPLs~\cite{Apel2013FOSPL,pohl2005software}.
Configurable features in an SPL are managed through variability mechanisms (e.g., variability models, such as feature models) to organize, implement, and document features with their dependencies~\cite{Kang1990FODA,Apel2013FOSPL,Czarnecki2012CoolFeatures,Schaefer2012SoftwareDiversity,Nesic2019Principles}.
Such mechanisms are typically supported by tools that check whether a configuration is valid, propagate configuration decisions~\cite{Krieter2016ModelSlicing,Apel2013FOSPL}, and derive valid variants automatically (e.g., FeatureIDE~\cite{Meinicke2017FeatureIDE}, pure::variants~\cite{Beuche2012purevariants}).
Consequently, configurable systems can be defined using the definitions of problem space (i.e., a domain abstraction), solution space (i.e., the implementation), and a mapping between both spaces (i.e., connection between both spaces allowing to derive variants automatically)~\cite{Apel2013FOSPL}.
Moreover, every configurable system can be verified based on certain attributes.
For this purpose, three established strategies exist: feature-based (i.e., analyzing each feature in isolation without considering configurations or dependencies), product-based (i.e., analyzing a system configuration based on its code or an abstraction), and family-based (i.e., analyzing the whole configurable system including valid configurations)~\cite{Thuem2014AnalysisSurvey}.


\subsection{Data Storages}\label{sec:background:storage}

A data storage, a medium that is able to store specific data in a certain way, is usually classified according to one of three structures it employs, namely centralized, decentralized, or distributed~\cite{furht2010CloudFundamentals,wu2017distributed}. 
Centralized storages (e.g., centralized relational SQL databases) are built around one single unit handling all major processing or storage tasks (e.g., one server).
Consequently, all machines are connected to the central unit where the data is stored~\cite{ramamritham1996RealTimeDatabases,chowdhury2018BlockchainAnalysis}.
In contrast, decentralized storages (e.g., decentralized NOSQL databases in the context of blockchain applications) rely on the distribution of processing or storage steps among several units with no or limited coordination~\cite{wu2017distributed,ayoade2018DecentralizedEnvironment}.
So, the dependency on an individual processing unit is much weaker than in centralized solutions~\cite{ayoade2018DecentralizedEnvironment,mcconaghy2016BigchainDB:Database,Bao2020WhenIssues}.
If a decentralized storage provides (close) coordination between independent units, it is called a distributed storage~\cite{van2002distributed,wu2017distributed}.
For instance, inter-cloud environments relying on a variety of storages are usually associated with this structure~\cite{celesti2010security,leite2016autonomic}. 

Data storages usually operate in certain environments that are oriented towards self-hosting (i.e., an environment located on a local machine) or outsourcing.
In the context of outsourcing, which has become increasingly widespread for enterprises in recent years, there are several common technologies, such as cloud, edge, and fog computing~\cite{gabel2017secure,correia2020SafeguardingEdge}.
Such environments use a complex combination of software and hardware components, and are able to provide server-based storage space at a high level of efficiency, flexibility, scalability, and on-demand availability~\cite{markovic2013SmartComputing,silva2018SecurityComputing}.  
Outsourcing solutions often serve as an underlying technology based on which data storing and processing functionalities can be implemented, leading to service-based systems or infrastructures; usually called SaaS or Infrastructure as a Service (IaaS)~\cite{lee2009quality,kulkarni2012cloud,serrano2015infrastructure}.


\subsection{Security}\label{sec:background:security}

According to norms and standards established in practice (e.g., ISO/IEC 27000 series~\cite{ISOIEC270002018}, ISO/IEC 25010~\cite{ISOIEC250102011}, ISO/IEC 29100~\cite{ISOIEC291002011}, NIST Guide for Conducting Risk Assessments~\cite{guidegonductingrisk2012}), security is a property of a software system aimed at protecting stored or processed data against a wide range of threats caused by attacks, vulnerabilities, errors (i.e., bugs), or nature (e.g., in the context of hardware).
A threat is defined as an unwanted, but possible, event that results in a harm to a system.
If there is a concrete possibility of exploiting such a threat, for instance, in terms of vulnerabilities, the threat poses a security risk~\cite{ISOIEC270002018,guidegonductingrisk2012}.
To minimize such risks, a regular risk assessment using defined monitoring, measurement, and analysis processes according to the data lifecycle of the individual system is essential~\cite{ISOIEC270012013,ISOIEC270042016}.
Furthermore, risks are also reduced by fulfilling defined security goals implemented via mitigation techniques, such as cryptographic control mechanisms~\cite{ISOIEC270002018,ISOIEC270012013,guidegonductingrisk2012}.
According to the ISO/IEC 27000 series~\cite{ISOIEC270002018} and ISO/IEC 25010~\cite{ISOIEC250102011}, three security goals are particularly important, which are also known as the CIA triad~\cite{lundgren2019defining,samonasCIAStrikesBack2014}:
\begin{itemize}[nosep,leftmargin=*]
	\item\emph{Confidentiality:} The data is only available to authorized entities.
	
	\item\emph{Integrity:} The data can only be modified by an authorized entity.
	
	\item\emph{Availability:} The system ensures timely and reliable access to its data for authorized entities.
\end{itemize}
In the context of information security, these three goals are usually extended by~\cite{ISOIEC250102011,ISOIEC270002018,guidegonductingrisk2012}:
\begin{itemize}[nosep,leftmargin=*]
	\item\emph{Accountability:} Any action can be traced to a unique entity.
    
	\item\emph{Authenticity:} The identity of an entity can be clearly proven to be the one claimed.
	
	\item\emph{Non-repudiation:} It is possible to prove the occurrence of every action and what entities were involved.
\end{itemize}
Information security is often associated with further security goals, such as reliability or privacy~\cite{ISOIEC270002018,ISOIEC291002011}.
However, such goals are usually subordinated to the six we described, have particular legal dependencies (e.g., legal regulations such as GDPR or HIPAA~\cite{tovino2017hipaa}), or their fulfillment is significantly related to or supported by the fulfillment of the six goals.

		

\section{Methodology} \label{sec:methodology}

The objective of this study was to identify, classify, and discuss research on configurable storage systems and security.
To achieve our objective, we conducted a systematic mapping study based on the guidelines of~\citet{Kitchenham2015EvidenceSLR}.
The individual steps of the methodology are presented in the following.

\subsection{Initial Screening}

In the first step, we employed an initial screening to ensure the need for our research, i.e., there are no recent similar studies and a sufficient and relevant body-of-knowledge.
Therefore, we searched the literature databases \textsc{Scopus}\footnote{www.scopus.com}, \textsc{IEEE Xplore}\footnote{ieeexplore.ieee.org}, and the \textsc{ACM Guide to Computing Literature}\footnote{dl.acm.org/search/advanced} by using the following search string without any constraints except the thematic focus (e.g., a certain time frame).

\searchString{"security" AND "software product line"} 

Based on the general search string, we found around 1,000 results emphasizing the relevance and research interests regarding security in the context of configurable systems. 
We note that our previous mapping study on safety and security for configurable systems~\citep{Kenner2021MappingStudy} only focused on configurable software systems but did not considered configurable storage systems in a certain way. 
Thus, our research objectives can be considered as valuable enough for our systematic mapping study focusing on security and configurable storage systems.  

\subsection{Study Design}

Based on the identified research interest, we conducted the systematic mapping study.
Precisely, we focused on security and configurable storage systems based on SPL.
We tried to cover a wide range of different storage systems, without any restriction regarding their architecture (e.g., centralized or decentralized), including solutions ranging from databases to service-based cloud storage.
So, we defined the following search string:

\searchString{("software as a service" OR "SaaS" OR "infrastructure as a service" OR "IaaS" OR "service-based" OR "service-oriented" OR "on-demand" OR "stor*" OR "cloud" OR "database" OR "warehouse" OR "blockchain" OR "edge" OR "fog") AND ("software product line" OR "SPL" OR "product famil*" OR "system famil*" OR “software famil*” OR "variant*rich" OR "config* system") AND ("security" OR "secure")}

Based on the search we employed an automated search on \textsc{Scopus}, \textsc{IEEE Xplore}, and the \textsc{ACM Guide to Computing Literature}.
These literature databases ensure a certain quality by indexing only peer-reviewed publications from a variety of publishers.
This way, we also reduced the threat of missing highly relevant publications.




\section{Results} \label{sec:results}

In this section, we describe the data extracted from the 50 selected papers.
Note that this section is generally structured based on the four thematic categories (i.e., publication, storage, security, and configurable system) as well as the individual extraction criteria.
In Table XYZ, an overview of the results is provided.

\subsection{Publication}

First, we describe the results which a more connected to the publications, i.e. typical publication data in terms of the publications years and indicated trends based on them, the domains, and the publications' perspectives regarding storage, security, and configurable systems. 

\myPar{Publication Years}

\myPar{Domains}

\myPar{Perspectives}

\subsection{Storage}

\myPar{Type of Storage}

\myPar{Organization Structure}

\myPar{Stored Data}

\subsection{Security}

\myPar{Specification}

\myPar{Standards}

\myPar{Threats}

\myPar{Goals}

\myPar{Mitigation Techniques}

\subsection{Configurable System}

\myPar{Variability Focus}

\myPar{Projection}

\myPar{Verification}

\myPar{Evolution}

\myPar{Tool Support}

\bgroup
\rowcolors{0}{white}{gray!25}
\begin{table*}[t]
	\caption{\todo{ggf. 2 Tabellen draus machen}}
	\label{tab:security_main}
	\vspace*{-2ex}
	\centering \begin{adjustbox}{max width = 0.75\textwidth}

	\begin{tabular}{c*{28}{C{\fixedColWidth}}}
		\hiderowcolors
		& & & & & & & & & & & \multicolumn{6}{c}{Security goals} & & & & & & & & & \\\cmidrule[0.4pt](r{0.2em}l{0.1em}){12-17}%
		 &
		 &
		\multicolumn{4}{c}{\tabrotate{Perspectives}} &
		\multicolumn{4}{c}{\tabrotate{Storage}}
		&
		&
	    \multicolumn{3}{c}{\tabrotate{CIA triad}} &
		\multicolumn{3}{c}{\tabrotate{\parbox{1.7cm}{Information security}}} &
		&
		&
		&
		&
		\multicolumn{3}{c}{\tabrotate{Projection}} &
		\multicolumn{3}{c}{\tabrotate{Verification}} &
		&
		\\
		\cmidrule[0.4pt](r{0.2em}l{0.1em}){3-6}%
		\cmidrule[0.4pt](r{0.2em}l{0.1em}){7-10}%
		\cmidrule[0.4pt](r{0.2em}l{0.1em}){12-14}%
		\cmidrule[0.4pt](r{0.2em}l{0.1em}){15-17}%
		\cmidrule[0.4pt](r{0.2em}l{0.1em}){22-24}%
		\cmidrule[0.4pt](r{0.2em}l{0.1em}){25-27}%
		
		\tabrotate{Reference} & 
		\tabrotate{Domain} &
		\tabrotate{Security $\rightarrow$ SPL} & 
		\tabrotate{SPL $\rightarrow$ Security} & 
		\tabrotate{Storage $\rightarrow$ SPL} & 
		\tabrotate{SPL $\rightarrow$ Storage} & 
		\tabrotate{Type of storage} & 
		\tabrotate{Organization structure} & 
		\tabrotate{Stored data type} & 
		\tabrotate{Data access} & 
		\tabrotate{Standard} & 
		\tabrotate{Confidentiality} & 
		\tabrotate{Integrity} & 
		\tabrotate{Availability} & 
		\tabrotate{Authorization} & 
		\tabrotate{Accountability} & 
		\tabrotate{Non-repudiation} & 
		\tabrotate{Specification} &
		\tabrotate{Security threats} &
		\tabrotate{Security technology} &
		\tabrotate{Variability focus} &
		\tabrotate{Problem space} & 
		\tabrotate{Solution space} & 
		\tabrotate{Mapping} & 
		\tabrotate{Product-based} & 
		\tabrotate{Family-based} & 
		\tabrotate{Feature-based} & 
		\tabrotate{Evolution} &
		\tabrotate{Tool support}
		\\
		\toprule
		\showrowcolors
		\cite{benlachgar2013review} & & & & & & & & & & & & & & & & & & & & & & & & & & & & \\
		\cite{galster2013constraints} & & & & & & & & & & & & & & & & & & & & & & & & & & & & \\
		\cite{marinho2013mobiline} & & & & & & & & & & & & & & & & & & & & & & & & & & & & \\
		\cite{gherardi2014software} & & & & & & & & & & & & & & & & & & & & & & & & & & & & \\
		\cite{matar2014towards} & & & & & & & & & & & & & & & & & & & & & & & & & & & & \\
		\cite{moens2014feature} & & & & & & & & & & & & & & & & & & & & & & & & & & & & \\
		\cite{moens2014cost} & & & & & & & & & & & & & & & & & & & & & & & & & & & & \\
		\cite{nguyen2014exploring} & & & & & & & & & & & & & & & & & & & & & & & & & & & & \\
		\cite{parra2014soa} & & & & & & & & & & & & & & & & & & & & & & & & & & & & \\
		\cite{walraven2014efficient} & & & & & & & & & & & & & & & & & & & & & & & & & & & & \\
		\cite{azzolini2015evolving} & & & & & & & & & & & & & & & & & & & & & & & & & & & & \\
		\cite{baresi2015dynamically} & & & & & & & & & & & & & & & & & & & & & & & & & & & & \\
		\cite{brisaboa2015reusable} & & & & & & & & & & & & & & & & & & & & & & & & & & & & \\
		\cite{galindo2015supporting} & & & & & & & & & & & & & & & & & & & & & & & & & & & & \\
		\cite{fernandez2015patterns} & & & & & & & & & & & & & & & & & & & & & & & & & & & & \\
		\cite{fernandez2015cloud} & & & & & & & & & & & & & & & & & & & & & & & & & & & & \\
		\cite{fernandez2015patterns} & & & & & & & & & & & & & & & & & & & & & & & & & & & & \\
		\cite{garcia2015software} & & & & & & & & & & & & & & & & & & & & & & & & & & & & \\
		\cite{moens2015allocating} & & & & & & & & & & & & & & & & & & & & & & & & & & & & \\
		\cite{tizzei2015architecting} & & & & & & & & & & & & & & & & & & & & & & & & & & & & \\
		\cite{van2015variability} & & & & & & & & & & & & & & & & & & & & & & & & & & & & \\
		\cite{cao2015constraint} & & & & & & & & & & & & & & & & & & & & & & & & & & & & \\
		\cite{ali2016requirements} & & & & & & & & & & & & & & & & & & & & & & & & & & & & \\
		\cite{dig2016cope} & & & & & & & & & & & & & & & & & & & & & & & & & & & & \\
		\cite{jumagaliyev2016Evolving} & & & & & & & & & & & & & & & & & & & & & & & & & & & & \\
		\cite{leite2016autonomic} & & & & & & & & & & & & & & & & & & & & & & & & & & & & \\
		\cite{metzger2016coordinated} & & & & & & & & & & & & & & & & & & & & & & & & & & & & \\
		\cite{passos2016coevolution} & & & & & & & & & & & & & & & & & & & & & & & & & & & & \\
		\cite{perrouin2016complexity} & & & & & & & & & & & & & & & & & & & & & & & & & & & & \\
		\cite{preuveneers2016systematic} & & & & & & & & & & & & & & & & & & & & & & & & & & & & \\
		\cite{preuveneers2016feature} & & & & & & & & & & & & & & & & & & & & & & & & & & & & \\
		\cite{syed2016cloud} & & & & & & & & & & & & & & & & & & & & & & & & & & & & \\
		\cite{arrieta2015cyber}	& & & & & & & & & & & & & & & & & & & & & & & & & & & & \\
		\cite{alferez2017achieving} & & & & & & & & & & & & & & & & & & & & & & & & & & & & \\
		\cite{jalil2017adapting} & & & & & & & & & & & & & & & & & & & & & & & & & & & & \\
		\cite{leite2017dohko} & & & & & & & & & & & & & & & & & & & & & & & & & & & & \\
		\cite{mohamed2017integrated} & & & & & & & & & & & & & & & & & & & & & & & & & & & & \\
		\cite{munoz2017green} & & & & & & & & & & & & & & & & & & & & & & & & & & & & \\
		\cite{khan2017variability} & & & & & & & & & & & & & & & & & & & & & & & & & & & & \\
		\cite{butting2018controlled} & & & & & & & & & & & & & & & & & & & & & & & & & & & & \\
		\cite{krieter2018towards} & & & & & & & & & & & & & & & & & & & & & & & & & & & & \\
		\cite{aouzal2019policy} & & & & & & & & & & & & & & & & & & & & & & & & & & & & \\
		\cite{krieter2019using} & & & & & & & & & & & & & & & & & & & & & & & & & & & & \\
		\cite{shaaban2019ontology} & & & & & & & & & & & & & & & & & & & & & & & & & & & & \\
		\cite{varela2019cyberspl} & & & & & & & & & & & & & & & & & & & & & & & & & & & & \\
		\cite{assunccao2020variability} & & & & & & & & & & & & & & & & & & & & & & & & & & & & \\
		\cite{siegmund2020dimensions} & & & & & & & & & & & & & & & & & & & & & & & & & & & & \\
		\cite{varela2020definition} & & & & & & & & & & & & & & & & & & & & & & & & & & & & \\
		\cite{varela2021carmen} & & & & & & & & & & & & & & & & & & & & & & & & & & & & \\
		\cite{zhang2021evolutionary} & & & & & & & & & & & & & & & & & & & & & & & & & & & & \\
		\bottomrule
		\midrule[-2pt]			
	\end{tabular}
	\end{adjustbox}
\vspace*{-2ex}
\end{table*}
\egroup

\section{Discussion} \label{sec:discussion}


security and privacy is lagging behind the technological advances regarding configurable systems and their data storage. 
Problem in the context of data storage is not new~\cite{strohbach2016big}

\subsection{Threats to Validity}

\section{Related Work} \label{sec:relatedwork}

We found six related papers contributing literature reviews or mapping studies in the field of configurable systems also involving security or security.
However, security is usually only extracted or mentioned as one of several quality attributes or non-functional requirements from a high-level perspective, without a detailed analysis of security concerns (e.g., threats, attack mitigation techniques).
Moreover, data storages are typically not addressed in the context of configurable systems and security.
In contrast to the related work we found, our study provides a comprehensive and systematic overview of security in the context of configurable data storages, providing a detailed analysis of these aspects.  

\looseness=-1
\citet{myllarniemi2012systematically} conducted a literature review of 29 papers (2000--2010) focusing on the variability of quality attributes that are part of SPLs.
The authors considered security as a quality attribute from a high-level perspective.  
\citet{benlachgar2013review} reviewed four SPL models for SaaS applications (without any time restriction), including an assessment of the models' relevance.
However, they do not consider security aspects. 
\citet{mahdavi-hezavehiVariabilityQualityAttributes2013} report a literature review of 46 papers (2000--2011) focusing on the variability of quality attributes of service-based software systems, including general security goals.
The authors found that only a few papers actually consider security in a general way as quality attributes.
\citet{hammaniSurveyNonFunctionalRequirements2014b} surveyed non-functional requirements in nine papers in the context of modeling and verifying SPLs (without any time restriction).
However, they only provide a high-level overview regarding security without focusing on details of the security concerns or data storages.
\citet{geraldi2020software} reviewed 56 SPL-related papers (2006--2018) that describe concepts or applications in the context of the internet-of-things. 
As cloud systems are closely associated with the internet-of-things, they are considered as part of the study, but not examined in detail. 
Security is only generally considered from a high level in the context of non-functional requirements.
\citet{Kenner2021MappingStudy} presented a systematic mapping study of 65 papers (2011--2020) with a focus on safety and security for configurable software systems, which is the study most closely related to our own.
We complement this mapping study by providing a detailed analysis of security research on configurable data storages, a perspective not scratched in the study of \citeauthor{Kenner2021MappingStudy}.

\section{Conclusion} \label{sec:conclusion}

In this article, we presented a systematic mapping study to provide an overview understanding of SPL-related research on security in the context of configurable systems and their data storage. 
Precisely, 50 publications (2013--2022) of a variety of domains were analyzed.
Overall, we identified highly interesting insights and XY concrete research gaps.

\todo{add details}

For future research, we plan to build on the findings of this study, especially those related to security in the context of data storage or the communication between software system and storage.
In more detail, we aim to conduct a systematic literature to extract more detailed results. 
Moreover, it is planned to analyze the technological structure of configurable storage systems to develop a framework consisting of their most relevant technological layers as a basis for a derivation of configurable security patterns.





%\begin{acks}
%\end{acks}

\bibliographystyle{ACM-Reference-Format}
\bibliography{shortConf,publishers,references}

	
%\balance
\end{document}