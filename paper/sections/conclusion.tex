\section{Conclusion} \label{sec:conclusion}

\looseness=-1
In this paper, we reported a systematic mapping study of security in the context of configurable data storages.
Precisely, we reviewed 50 papers (2013--2022) from a variety of domains.
We provided key insights and 14 opportunities for future research.
Particularly, we emphasize that, despite the high relevance of security for configurable systems, little research has been concerned with configurable data storages---which store, manage, and provide access to potentially critical data of customers, the organization, or the system itself.
Generally, we are missing a detailed understanding of how the current research is related to established security mechanisms and established standards in practice.
As a result, the transfer of (theoretical) concepts into practice is impaired.
Furthermore, there is no uniform understanding of data storages, which challenges comparisons between papers. 
This issue can only be solved in the future through a uniform understanding of relevant terms and technological layers, as well as by consistently addressing security goals within configurable systems as features of both the system and the data storage.

In future work, we aim to expand on this study to improve our understanding of the concepts, processes, and relationships of configurable systems and data storages.
For instance, one objective is to assess the impact of the binding time, for instance, analyzing security-related differences of configurability at design time and runtime.
Moreover, we plan to analyze and compare the technological structures' and the different layers' impact on \enquote{regular} and configurable storages to develop techniques and security patterns supporting the engineering and evolution of configurable storages.



