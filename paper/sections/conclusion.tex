\section{Conclusion} \label{sec:conclusion}

In this article, we presented a systematic mapping study to provide an overview understanding of security in the context of configurable storages based on product-line engineering.
Precisely, 50 publications (2013--2022) of a variety of domains were analyzed.

Overall, we identified highly interesting insights and 16 research gaps.
We emphasize that despite the high relevance of security for configurable systems, only little research has been done, especially with regard to data storage that is actually configurable.
In general, there are few existing concepts, but they usually do not refer to concrete, detailed security mechanisms or international security standards. 
As a result, the transfer of (theoretical) concept into practice is heavily impaired.
Furthermore, there is no uniform understanding of storage, which makes comparable evaluations almost impossible to implement. 
This issue can only be solved in the future through a uniform definition of relevant terms and technological layers, as well as the consistent addressing of security requirements within the system features of both the software system and the storage system.
For future research, we plan to build on the findings of this study, especially those related to security in the context of data storage or the communication between software system and storage.
In more detail, we aim to conduct a systematic literature review of the security concepts we found as well as related concepts to extract more in-depth results. 
Moreover, it is planned to analyze the technological structure of configurable storages to develop a framework consisting of their most relevant technological layers as a basis for a derivation of variable security patterns.



