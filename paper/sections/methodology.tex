\section{Methodology} \label{sec:methodology}

Our objective for this study was to identify, classify, and discuss research in the intersection of configurable data storages and security.
To achieve this objective, we conducted a systematic mapping study based on the guidelines of~\citet{Kitchenham2015EvidenceSLR}.
In the following, we describe the individual steps of our study, for which we display an overview in \autoref{fig:methodology}.

\subsection{Initial Screening}

At first, we employed an initial screening to ensure the need for our study.
Precisely, we aimed to ensure that there are no recent studies addressing our objective, and that a sufficient and relevant number of papers exists.
Therefore, we searched the literature databases \textsc{Scopus}, \textsc{IEEE Xplore}, and the \textsc{ACM Guide to Computing Literature} using the following search string without any other constraints (e.g., a certain time frame):

\searchString{\enquote{software product line} and \enquote{security}} 
%
Based on this general search string, we obtained around 1,000 papers emphasizing the relevance and research interest for security in the context of configurable systems. 
Thus, we considered our research objective valuable for conducting a systematic mapping study.
We note that we are aware of the mapping study on safety and security for configurable systems by~\citet{Kenner2021MappingStudy}. 
However, our goals differ from this and other studies referenced therein, since we cover another body of research (cf. \autoref{sec:relatedwork}).
We focus on the intersection of security and data storages, which has not been the subject of previous mapping studies. 

\subsection{Study Design}

Building on the identified research interest, we decided to conduct a systematic mapping study.
Precisely, we focused on security and configurable data storages, also considering potentially underlying configurable software systems that may be the actual focus of a paper.
We considered keywords that cover a wide range of storage systems, without any restrictions regarding their architecture (e.g., centralized or decentralized), including solutions ranging from databases to service-based cloud storages.
Moreover, we built on the experiences we obtained during the initial screening and considered the previous studies we identified to derive synonyms.
So, we defined the following search string:

\searchString{(\enquote{software as a service} OR \enquote{SaaS} OR \enquote{infrastructure as a service} OR \enquote{IaaS} OR \enquote{service-based} OR \enquote{service-oriented} OR \enquote{on-demand} OR \enquote{stor*} OR \enquote{cloud} OR \enquote{database} OR \enquote{blockchain} OR \enquote{edge} OR \enquote{fog}) AND 
\newline
(\enquote{software product line} OR \enquote{product famil*} OR \enquote{system famil*} OR \enquote{software famil*} OR \enquote{variant*rich} OR \enquote{config* system}) AND 
\newline
(\enquote{security} OR \enquote{secure})}
%
This search string comprises relevant terms in the context of data storages, types, and properties (e.g., \enquote{database,} \enquote{SaaS,} \enquote{on-demand}), configurable systems in the context of SPLs (e.g., \enquote{software product line,} \enquote{system family}), and security in general (e.g., \enquote{secure})---since we assumed that more specific security concerns are rarely reported in the primary search fields.

Using this search string, we employed an automated search on \textsc{Scopus}, \textsc{IEEE Xplore}, and the \textsc{ACM Guide to Computing Literature}; searching within the typical fields title, abstract, and keywords.
These literature databases ensure a certain quality by indexing only peer-reviewed publications.
Moreover, \textsc{Scopus} and the \textsc{ACM Guide to Computing Literature} provide papers from a variety of publishers. 
This way, we reduced the threat of missing highly relevant publications from other publishers.
To further improve the completeness of our dataset, we additionally applied a backwards snowballing (i.e., analyzing the references of the already selected papers)~\cite{Wohlin2014Snowballing}.
Note that we performed only one iteration of snowballing to limit the effort of our study.


\begin{figure}
	\includegraphics[width=\linewidth]{img/methodology.pdf}
	\vspace*{-4ex}
	\caption{Methodological overview of our systematic mapping study. The numbers indicate the exact amount of papers that we considered relevant after the previous step.}
	\label{fig:methodology}
	\vspace*{-2ex}
\end{figure}

\subsection{Selection Criteria}

To identify and select relevant papers, we defined the following selection criteria:
\begin{enumerate}[label=IC\textsubscript{\arabic*},leftmargin=*,nosep]
    \item The paper is written in English.
    \item The paper has been published between 2013 and 2022.\label{it:icTime}
    \item The paper is a peer-reviewed conference paper or journal article (e.g., excluding keynote summaries or posters).\label{it:icPaper}
    \item The paper is longer than three pages.\label{it:icPages}
    \item The paper deals with security and configurable data storages in the context of SPLs.\label{it:icFocus}
\end{enumerate}
%
%\looseness=-1
We intentionally focused on the last decade (\ref{it:icTime}) of research to limit the number of papers and cover the most recent advancements in both research and practice.
Note that we did not perform a detailed quality assessment of all selected papers.
In fact, since our study was likely to cover a variety of research methods that cannot be properly compared against each other, common quality criteria are not applicable.
However, by defining a minimum number of pages (\ref{it:icPages}) and considering only peer-reviewed papers (\ref{it:icPaper}), we assume that each paper meets a certain quality that allows us to understand the addressed problem.
We intentionally excluded papers that do not address security and data storages in the context of SPLs (\ref{it:icFocus}), for instance, papers proposing a configurable SaaS system including a data storage or discussing security concerns, but without building on concepts from product-line engineering~\cite{liu2014towards,khan2019dynamically}.

\subsection{Data Extraction}\label{sec:extraction}

To extract data in a comprehensive and systematic way, we defined categories covering the two relevant areas security and configurable data storages.
We derived our extraction categories from related papers, particularly the related mapping study of \citet{Kenner2021MappingStudy} we identified during out initial screening and the papers referenced therein.
In more detail, we defined the categories based on concepts established in research on security as well as product-line engineering.
Particularly, the categories regarding security rely on established classifications of security goals (cf. \autoref{sec:background:security}).
However, we extended the criteria to gain more detailed insights, especially with respect to the intersection of security and configurable storages.
%Note that we do not extract typical publication data (e.g., publication communities, publication years) as we focus on a  qualitative, content-related analysis.

Overall, we defined the following extraction categories classified into four thematic topics:
\begin{enumerate}[nosep,leftmargin=*]
	\item \textbf{Publication}
	\begin{itemize}[nosep,leftmargin=*]
		\item \emph{Publication year} of a paper.
		\item \emph{Domain} of the research (e.g., production, automotive).
		\item \emph{Perspective} of a paper (indicated by an arrow ($\rightarrow$), which reads as \enquote{employed to}):
		\begin{itemize}[nosep,leftmargin=*]
			\item \emph{Security $\rightarrow$ SPL}: the paper focuses on the security of a configurable system.
			\item \emph{SPL $\rightarrow$ security}: the paper focuses on a security concern based on product-line concepts.
			\item \emph{Storage $\rightarrow$ SPL}: the paper focuses on a data storage used to support certain SPL tasks (e.g., storing variants).
			\item \emph{SPL $\rightarrow$ Storage}: the paper focuses on a storage system based on product-line concepts.
		\end{itemize}
	\end{itemize}

	\item \textbf{Storage}	
	\begin{itemize}[nosep,leftmargin=*]
		\item \emph{Type of storage medium} a paper is concerned with (e.g., a relational database in a SaaS cloud environment~\cite{curino2011relational}).
		\item \emph{Structural organization} of the proposed storage system (i.e., centralized, decentralized, distributed~\cite{hugoson2007centralized,wu2017distributed}).
		\item \emph{Stored data} indicating whether data is (securely) stored (e.g., feature models~\cite{shaaban2019ontology}, source code~\cite{leite2017dohko}).
	\end{itemize}

	\item \textbf{Security}	  
	\begin{itemize}[nosep,leftmargin=*]
		\item \emph{Standard} a publication refers to (e.g., standards of the ISO/IEC 27000 series, such as ISO/IEC 27001~\cite{ISOIEC270012013}, ISO/IEC 27002~\cite{ISOIEC270022013}).
		\item \emph{Security goals} a publications aims to achieve~\cite{anderson2003we,guidegonductingrisk2012,samonasCIAStrikesBack2014}: 
		\begin{itemize}[nosep,leftmargin=*]
			\item \emph{CIA triad:} \emph{confidentiality}, \emph{integrity}, and \emph{availability}.
			\item \emph{Information security:} \emph{authorization}, \emph{accountability}, and \emph{non-repudiation}.
		\end{itemize}
		\item \emph{Specification}, indicating how certain security goals are considered, documented, or managed (e.g., as quality attribute~\cite{peruma2018security} or non-functional requirements~\cite{ragab2015software}).
		\item \emph{Security threats} a paper is concerned with (e.g., credential reuse or SQL injection attacks~\cite{humayun2020cyber}).
		\item \emph{Mitigation techniques} proposed or implemented to mitigate the impact of cyber attacks and to achieve certain security goals (e.g., encryption techniques, such as AES~\cite{varela2021carmen}, or technologies, such as Intel SGX~\citep{Costan2016IntelExplained}).
	\end{itemize}

	\item \textbf{Configurable system}	  
	\begin{itemize}[nosep,leftmargin=*]	
		\item \emph{Variability focus}, indicating whether the storage itself is configurable or only the surrounding software system.
		\item \emph{Projection} of an SPL considered in the paper, namely problem space, solution space, or the mapping between both~\cite{Apel2013FOSPL}.
		\item \emph{Verification}, indicating whether the described method follows a product-, feature-, or family-based analysis strategy~\cite{Thuem2014AnalysisSurvey}.
		\item \emph{Evolution}, indicating whether storage evolution is considered in the paper~\cite{bosch2002maturity}.
		\item \emph{Tool support} reported in the paper (e.g., FeatureIDE~\cite{Meinicke2017FeatureIDE}).
	\end{itemize}
\end{enumerate}
Based on this data, we aim to synthesize a detailed overview on configurable data storages.
We remark that when we organized and synthesized the data (cf. \autoref{sec:conduct}), we found that many of the data fields were highly diverse and would obfuscate the presentation in \autoref{tab:security}.
For this reason, we provide a more concise overview in the table to focus on our core analysis, and publish a more detailed overview of the individual entries for each paper in our dataset.\textsuperscript{\ref{ft:repo}}

\subsection{Conduct}\label{sec:conduct}

The first author of this paper conducted the automated search on March \nth{15}, 2022, resulting in a total of 538 papers (65 from \textsc{Scopus}, 122 from \textsc{IEEE Xplore}, and 351 from the \textsc{ACM Guide to Computing Literature}).
After integrating the found papers into the literature review tool Rayyan QCRI,\footnote{\url{www.rayyan.ai}} the first author labeled each paper as included, excluded, or duplicate.
In this context, we conducted a duplicate removal (i.e., removing 29 papers) and employed our selection criteria on titles and abstracts, resulting in 76 papers for the full-text analysis. 
After reading the full-texts, we had to remove 32 more papers, since they only superficially dealt with configurable data storages or mentioned these aspects only in the context of related work.
Then, we employed one iteration of backwards snowballing on the remaining 44 papers.
Finally, we identified 50 papers to be relevant for our objectives, and we started with extracting data based on the defined categories.
For this data extraction, we used an Excel spreadsheet and performed an open-coding-like process to identify concrete instances of data fitting to our data.
Then, we employed an open-card-sorting-like methodology to classify recurring information and synthesize common themes for each data category.
