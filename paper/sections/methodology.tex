\section{Mapping Study Design} \label{sec:methodology}

The objective of this study was to identify, classify, and discuss research on configurable storage systems and security.
To achieve our objective, we conducted a systematic mapping study based on the guidelines of~\citet{Kitchenham2015EvidenceSLR}.
The individual steps of the methodology are presented in the following.

\subsection{Initial Screening}

In the first step, we employed an initial screening to ensure the need for our research, i.e., there are no recent similar studies and a sufficient and relevant body-of-knowledge.
Therefore, we searched the literature databases \textsc{Scopus}\footnote{www.scopus.com}, \textsc{IEEE Xplore}\footnote{ieeexplore.ieee.org}, and the \textsc{ACM Guide to Computing Literature}\footnote{dl.acm.org/search/advanced} by using the following search string without any constraints except the thematic focus (e.g., a certain time frame).

\searchString{"security" AND "software product line"} 

Based on the general search string, we found around 1,000 results emphasizing the relevance and research interests regarding security in the context of configurable systems. 
We note that our previous mapping study on safety and security for configurable systems~\citep{Kenner2021MappingStudy} only focused on configurable software systems but on configurable storage systems. 
Thus, our research objectives can be considered as valuable enough to initiate the systematic mapping study focusing on security and configurable storage systems.  

