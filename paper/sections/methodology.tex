\section{Methodology} \label{sec:methodology}

The objective of this study was to identify, classify, and discuss research on SPL-related storage systems and security.
To achieve our objective, we conducted a systematic mapping study based on the guidelines of~\citet{Kitchenham2015EvidenceSLR}.
The individual steps of the methodology are presented in the following.

\subsection{Initial Screening}

In the first step, we employed an initial screening to ensure the need for our research, i.e., there are no recent similar studies and a sufficient and relevant body-of-knowledge.
Therefore, we searched the literature databases \textsc{Scopus}, \textsc{IEEE Xplore}, and the \textsc{ACM Guide to Computing Literature} by using the following search string without any constraints except the thematic focus (e.g., a certain time frame).

\searchString{"software product line" and "security"} 

Based on the general search string, we found around 1,000 results emphasizing the relevance and research interests regarding security in the context of configurable systems. 
We note that our previous mapping study on safety and security for configurable systems~\citep{Kenner2021MappingStudy} only focused on configurable software systems but did not considered storage systems in a certain way. 
Thus, our research objectives can be considered as valuable enough for our systematic mapping study focusing on security and SPL-related storage systems.  

\subsection{Study Design}

Based on the identified research interest, we conducted the systematic mapping study.
Precisely, we focused on security and SPL-related storage systems, including the usually underlying configurable software systems.
We tried to cover a wide range of different storage systems, without any restriction regarding their architecture (e.g., centralized or decentralized), including solutions ranging from databases to service-based cloud storage.
So, we defined the following search string:

\searchString{("software as a service" OR "SaaS" OR "infrastructure as a service" OR "IaaS" OR "service-based" OR "service-oriented" OR "on-demand" OR "stor*" OR "cloud" OR "database" OR "warehouse" OR "blockchain" OR "edge" OR "fog") AND ("software product line" OR "product famil*" OR "system famil*" OR “software famil*” OR "variant*rich" OR "config* system") AND ("security" OR "secure")}

Based on the search we employed an automated search on \textsc{Scopus}, \textsc{IEEE Xplore}, and the \textsc{ACM Guide to Computing Literature}.
These literature databases ensure a certain quality by indexing only peer-reviewed publications from a variety of publishers.
This way, we reduced the threat of missing highly relevant publications, too.

\subsection{Selection Criteria}

For the identification and selection of relevant and suitable publications, we defined the following selection criteria:

\begin{enumerate}[label=IC\textsubscript{\arabic*},leftmargin=*,nosep]
    \item The publication is written in English.
    \item The publication has been published between 2013 and 2022.\label{it:icTime}
    \item The publication is a peer-reviewed conference paper or journal article (e.g., excluding keynote summaries or posters).\label{it:icPaper}
    \item The publication is longer than three pages.\label{it:icPages}
    \item The publication deals with security and SPL-related storage systems.\label{it:icFocus}
\end{enumerate}\medskip

We intentionally focused on the last decade (\ref{it:icTime}) to cover the most recent advancements in both research and practice.
In this context, we argue that publications older than ten years have usually already become established practices,  standards, or are no longer relevant.
Since we did not perform a detailed quality assessment of all selected papers, we tried to ensure a certain quality by relying on the review of publication venues by \textsc{Scopus}, \textsc{IEEE}, and \textsc{ACM} (\ref{it:icPaper}).
Moreover, by defining a minimum number of pages (\ref{it:icPages}), we assumed that the papers meeting this criterion provide enough details to understand the addressed problem.
We intentionally excluded approaches that do not address security SPL-related storage systems based on SPL (\ref{it:icFocus}).


\subsection{Data Extraction}

To extract data in a comprehensive and comparable way, we defined criteria covering the two relevant areas of security and product-line engineering.
We based our criteria on the extraction criteria of our previous mapping study on safety and security of configurable software systems~\citep{Kenner2021MappingStudy} as the topic was basically similar to our study.
In more detail, these criteria are derived from research and concepts on security or product-line engineering
So, the criteria regarding security rely on established classifications of security goals.
However, we extended the criteria to gain more detailed insights, especially regarding the intersection of product-line engineering and storage systems.
Note that we do not extract typical publication data (e.g., publication communities, publication years) as we focus on a  qualitative, content-related analysis.
Overall, we defined the following extraction criteria:
\begin{itemize}[nosep]
    \item \textbf{Perspective} of a publication (indicated by an arrow ($\rightarrow$) read as \enquote{employed to}):
    \begin{itemize}
	    \item \emph{Security $\rightarrow$ SPL}: publications focusing on the security of a configurable system
		\item \emph{SPL $\rightarrow$ security}: publications addressing a security concern based on product-line techniques
		\item \emph{Storage $\rightarrow$ SPL}: publications focusing on storage systems used to support certain SPL tasks (e.g., storing variants)
		\item \emph{SPL $\rightarrow$ Storage}: publications focusing on storage systems based on product-line techniques
	\end{itemize}
	\item \textbf{Domain} of a solution (e.g., manufacturing, automotive)
	\item \textbf{Type of storage} a publication is concerned with (e.g., database, cloud)
	\item \textbf{Organization structure} of the proposed storage system (i.e., centralized or decentralized)
	\item \textbf{Stored data type} indicating whether or which data is (securely) stored in a storage system
	\item \textbf{Security standard} a publication refers to (e.g., ISO 27001)
	\item \textbf{Security goals} a publications aims to achieve: 
	\begin{itemize}
	    \item \emph{CIA triad}, including confidentiality, integrity, and availability
	    \item \emph{Information security}, including authorization, accountability, and non-repudiation
	\end{itemize}
	\item \textbf{Specification} indicating how certain security concerns are considered, documented, or managed (e.g., as security goals or non-functional requirements)
	\item \textbf{Security threats} a publication is concerned with (e.g., concrete system vulnerabilities)
	\item \textbf{Security technology} implemented or proposed to achieve certain security goals (e.g., Intel Software Guard Extensions~\citep{Costan2016IntelExplained})
	\item \textbf{Variability focus} indicating whether the storage is actually configurable or only the underlying software system
	\item \textbf{Projection} of the product-line indicating whether the problem space, solution space, or the mapping between both is covered~\cite{Apel2013FOSPL}.
	\item \textbf{Verification} method indicating whether the described system is analyzed based on products, features, or the whole product family~\cite{Thuem2014AnalysisSurvey}.
	\item \textbf{Evolution} of a SPL-related storage system
	\item \textbf{Tool support} reported in the publication (e.g., FeatureIDE~\cite{Meinicke2017FeatureIDE}, pure::variants~\cite{Beuche2012purevariants}, own prototypes).
	
\end{itemize}

\subsection{Conduct}

The automated search was conducted on March 15, 2022, resulting in a total of 538 publications (65 from \textsc{Scopus}, 122 from \textsc{IEEE Xplore}, and 351 from the \textsc{ACM Guide to Computing Literature}).
After a duplication removal as well as a title and abstract selection, we considered 76 papers as suitable for the full-text selection.
Finally, 53 publications out of all papers were selected after reading the full-texts.

		
		
		
		
		
		

	
	
		
		
		
		
		

