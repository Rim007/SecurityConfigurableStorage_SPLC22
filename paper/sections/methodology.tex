\section{Methodology} \label{sec:methodology}

The objective of this study was to identify, classify, and discuss research on SPL-related storage systems and security.
To achieve our objective, we conducted a systematic mapping study based on the guidelines of~\citet{Kitchenham2015EvidenceSLR}.
The individual steps of the methodology are presented in the following.

\subsection{Initial Screening}

In the first step, we employed an initial screening to ensure the need for our research, i.e., there are no recent similar studies and a sufficient and relevant body-of-knowledge.
Therefore, we searched the literature databases \textsc{Scopus}, \textsc{IEEE Xplore}, and the \textsc{ACM Guide to Computing Literature} by using the following search string without any constraints except the thematic focus (e.g., a certain time frame).

\searchString{"software product line" and "security"} 

Based on the general search string, we found around 1,000 results emphasizing the relevance and research interests regarding security in the context of configurable systems. 
Thus, our research objectives can be considered as valuable enough for our systematic mapping study focusing on security and SPL-related storage systems.
We note that we are aware of the mapping study on safety and security for configurable systems by~\citet{Kenner2021MappingStudy}. 
However, our goals differ from this study and we cover another body of literature (i.e., papers) without focusing on safety. 

\todo{would add a concrete goal/objective subsection or at least bullet points. what is the actual intention of the study, why is it really necessary to understand the state of the art to what extent? just getting an overview for the sake of it is also okish, but I don't think that the topic is self-motivational enough and it will be criticized if there is no good motivation on why exactly this study is needed and what its concrete goals are}

\subsection{Study Design}

Based on the identified research interest, we conducted the systematic mapping study.
Precisely, we focused on security and SPL-related storage systems, including the usually underlying configurable software systems. \todo{as said, the motivation is not clear enough to me yet}
We considered keywords that cover a wide range of different storage systems, without any restriction regarding their architecture (e.g., centralized or decentralized), including solutions ranging from databases to service-based cloud storage.
So, we defined the following search string:

\searchString{("software as a service" OR "SaaS" OR "infrastructure as a service" OR "IaaS" OR "service-based" OR "service-oriented" OR "on-demand" OR "stor*" OR "cloud" OR "database" OR "blockchain" OR "edge" OR "fog") AND ("software product line" OR "product famil*" OR "system famil*" OR “software famil*” OR "variant*rich" OR "config* system") AND ("security" OR "secure")}


The search string consists of relevant terms in the context of data storage, relevant types, and properties (e.g., "database", "SaaS", or "on-demand"), configurable systems in the context of software product-line engineering (e.g., "software product line" or "system family"), and security in general as we assume that more specific concerns will only rarely be reported (e.g., "secure").

Based on the search string we employed an automated search on \textsc{Scopus}, \textsc{IEEE Xplore}, and the \textsc{ACM Guide to Computing Literature}.
These literature databases ensure a certain quality by indexing only peer-reviewed publications.
Moreover, \textsc{Scopus} and the \textsc{ACM Guide to Computing Literature} provide papers from a variety of publishers. 
This way, we reduced the threat of missing highly relevant publications, too.

\subsection{Selection Criteria}

For the identification and selection of relevant and suitable publications, we defined the following selection criteria:

\begin{enumerate}[label=IC\textsubscript{\arabic*},leftmargin=*,nosep]
    \item The publication is written in English.
    \item The publication has been published between 2013 and 2022.\label{it:icTime}
    \item The publication is a peer-reviewed conference paper or journal article (e.g., excluding keynote summaries or posters).\label{it:icPaper}
    \item The publication is longer than three pages.\label{it:icPages}
    \item The publication deals with security and SPL-related storage systems.\label{it:icFocus} \todo{the notion of SPL related storage system still confuses me xD}
\end{enumerate}\medskip

We intentionally focused on the last decade (\ref{it:icTime}) to limit the number of publications and cover the most recent advancements in both research and practice.
A detailed quality assessment of all selected papers was not performed by us.
Since our work covers a variety of works that cannot be properly compared against each other, common quality criteria are not applicable in our case.
However, by defining a minimum number of pages (\ref{it:icPages}), it is assumed that the papers meeting this criterion provide enough details to understand the addressed problem.
We intentionally excluded papers that do not address security, product-line engineering, or a data storage medium (\ref{it:icFocus}), e.g., publications proposing actual configurable SaaS systems, including a storage medium or detailed security concerns but without building on product-line techniques~\cite{liu2014towards,khan2019dynamically}.

\subsection{Data Extraction}

To extract data in a comprehensive and comparable way, we defined criteria covering the two relevant areas of security and product-line engineering.
We based our criteria on the extraction criteria of our previous mapping study on safety and security of configurable software systems~\citep{Kenner2021MappingStudy} as the topic was basically similar to our study.
In more detail, these criteria are derived from research and concepts on security or product-line engineering
So, the criteria regarding security rely on established classifications of security goals.
However, we extended the criteria to gain more detailed insights, especially regarding the intersection of product-line engineering and storage systems.
%Note that we do not extract typical publication data (e.g., publication communities, publication years) as we focus on a  qualitative, content-related analysis.
Overall, we defined the following extraction criteria classified into four thematic categories:

\noindent\textbf{Publication:}
\begin{itemize}[nosep]
    \item \textbf{Publication year} of a paper.
    \item \textbf{Domain} of a solution (e.g., production, automotive).
    \item \textbf{Perspective} of a publication (indicated by an arrow ($\rightarrow$) read as \enquote{employed to}):
    \begin{itemize}
	    \item \emph{Security $\rightarrow$ SPL}: publications focusing on the security of a configurable system.
		\item \emph{SPL $\rightarrow$ security}: publications addressing a security concern based on product-line techniques.
		\item \emph{Storage $\rightarrow$ SPL}: publications focusing on storage systems used to support certain SPL tasks (e.g., storing variants).
		\item \emph{SPL $\rightarrow$ Storage}: publications focusing on storage systems based on product-line techniques.
	\end{itemize}
	\end{itemize}
	
\noindent\textbf{Storage:}	
    \begin{itemize}[nosep]
	\item \textbf{Type of storage medium} a publication is concerned with (e.g., a relational database in a SaaS cloud environment~\cite{curino2011relational}).
	\item \textbf{Organization structure} of the proposed storage system (i.e., centralized or decentralized~\cite{hugoson2007centralized}).
	\item \textbf{Stored data} indicating whether or which data is (securely) stored by a storage system (e.g., feature models~\cite{shaaban2019ontology}, source code~\cite{leite2017dohko})
    \end{itemize}
    
\noindent\textbf{Security:}	  
    \begin{itemize}[nosep]
	\item \textbf{Standard} a publication refers to (e.g., standards of the ISO/IEC 27000 series, such as ISO/IEC 27001~\cite{ISOIEC270012013}, ISO/IEC 27002~\cite{ISOIEC270022013}).
	\item \textbf{Security goals} a publications aims to achieve~\cite{anderson2003we,guidegonductingrisk2012,samonasCIAStrikesBack2014}: 
	\begin{itemize}
	    \item \emph{CIA triad}, including confidentiality, integrity, and availability.
	    \item \emph{Information security}, including authorization, accountability, and non-repudiation.
	\end{itemize}
	\item \textbf{Specification} indicating how certain security concerns are considered, documented, or managed (e.g., as quality attribute~\cite{peruma2018security} or non-functional requirements~\cite{ragab2015software}).
	\item \textbf{Security threats} a publication is concerned with (e.g., credential reuse or SQL injection attacks~\cite{humayun2020cyber}).
	\item \textbf{Mitigation techniques} proposed or implemented to mitigate the impact of cyber-attacks and to achieve certain security goals (e.g., encryption techniques such as AES~\cite{varela2021carmen} or concrete technologies Intel SGX~\citep{Costan2016IntelExplained}).
	\end{itemize}
	
\noindent\textbf{Configurable system:}	  
    \begin{itemize}[nosep]	
	\item \textbf{Variability focus} indicating whether the storage is actually configurable or only the underlying software system.
	\item \textbf{Projection} of the product-line indicating whether the problem space, solution space, or the mapping between both is covered~\cite{Apel2013FOSPL}.
	\item \textbf{Verification} method indicating whether the described system is analyzed based on products, features, or the whole product family~\cite{Thuem2014AnalysisSurvey}.
	\item \textbf{Evolution} of a SPL-related storage system~\cite{bosch2002maturity}.
	\item \textbf{Tool support} reported in the publication. (e.g., FeatureIDE~\cite{Meinicke2017FeatureIDE}, pure::variants~\cite{Beuche2012purevariants}).
\end{itemize}

\subsection{Conduct}
The automated search was conducted on March 15, 2022, resulting in a total of 538 publications (65 from \textsc{Scopus}, 122 from \textsc{IEEE Xplore}, and 351 from the \textsc{ACM Guide to Computing Literature}).
After integrating the selected papers into the literature review tool Rayyan QCRI\footnote{\url{www.rayyan.ai}} where the literature was labeled as included, excluded, or duplicate.
In this context, we conducted a duplication removal (i.e., removing 29 papers) as well as a title and abstract selection, resulting in 76 papers for the full-text selection. 
However, after reading the full-texts, we had to remove 32 paper since they only superficially dealt with SPL and data storage or only mentioned these aspects in the context of related work.
To increase the number of selected papers, we additionally applied a backward snowballing process (i.e., analyzing the references of the already selected literature).
Finally, 50 publications were selected for the extraction of data based on the defined criteria.
The data extraction was performed in an Excel spreadsheet in the context of an open coding-like process based category labels (i.e., identifying and classifying the information found numerous times). 
The methodological process including the exact numbers of publications is illustrated in Figure~\ref{fig:methodology}.
		
\todo{why 50 papers are enough?}
		
\begin{figure}
	\includegraphics[width=\linewidth]{img/methodology.pdf}
	\vspace*{-4ex}
	\caption{Methodological overview of the systematic mapping study (numbers indicate the exact amount of selected publications}
	\label{fig:methodology}
	\vspace*{-2ex}
\end{figure}
		
		

	
	
		
		
		
		
		

