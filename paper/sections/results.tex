\section{Results} \label{sec:results}

In this section, we describe the data extracted from the 50 selected papers.
Note that this section is generally structured based on the four thematic categories (i.e., publication, storage, security, and configurable system) as well as the individual extraction criteria.
In Table XYZ, an overview of the results is provided.

\subsection{Publication}

First, we describe the results which are connected to the publications, i.e. typical publication data in terms of the publications years and indicated trends based on them, the domains, and the publications' perspectives regarding storage, security, and configurable systems. 

\myPar{Publication Years}
The analysis of the publication data reveals two relevant trends regarding the research interest.
First, after three publications in 2013, the majority of the selected publications was published between 2014 and 2017 (in total 36 papers, on average nine papers per year). 
Second, between 2018 and 2021, the number of published papers decreases (in total 11 papers, on average three papers per year). 
We did not identify a paper published in 2022.
\todo{not sure about the relevance of the publications years in general...or maybe adding a figure showing the trend @Jacob what do you think?}

\myPar{Domains}
The publications cover a variety of domains. 
In 28 paper, the domain was explicitly mentioned where 22 papers only describe SPL-based approaches for universal or unspecified domains.  
Most publications (8) are concerned with the production domains, i.e. they usually describe configurable systems or configurations to support certain production systems, machines, or processes such as cyber-physical systems~\cite{arrieta2015cyber,varela2020definition}.
Moreover, we identified publications focusing on retail (8 papers, e.g., an ERP system~\cite{ali2016requirements}), administration and management services (5 papers, e.g., an eGovernment tool~\cite{galster2013constraints}), general web applications (4 papers, e.g., an online survey tool~\cite{jumagaliyev2016Evolving}), and payment (4 papers, e.g., an ePayment system~\cite{preuveneers2016feature}). 
Other domains which we found only once each include mobile services~\cite{marinho2013mobiline}, medical systems~\cite{moens2014cost}, geographic systems~\cite{brisaboa2015reusable}, tourism systems~\cite{benlachgar2013review}, and general Linux-based systems~\cite{passos2016coevolution}.

\myPar{Perspectives} 
Regarding the first perspective classification (security and SPL), we identified that most publications (37) are concerned with security for configurable systems (Security $\rightarrow$ SPL), e.g., an adaptive Software Guards extension-enabled configurable system~\cite{krieter2019using}.
Ten publications focus on the application of product-line techniques to realize certain security concerns (SPL $\rightarrow$ Security), e.g., security configurations of cyber-physical systems~\cite{varela2020definition}.
Moreover, we found three papers where a classification was hardly possible, as they consider both perspectives, e.g., a security policy-driven software for SaaS service configurations~\cite{aouzal2019policy}.
Regarding the second perspective classification (storage and SPL), 28 publications deal with storage for configurable systems (Storage $\rightarrow$ SPL), e.g., a configurable eCommerce software system including a database~\cite{azzolini2015evolving}.
In addition, in 18 papers, the perspective regarding the application of product-line techniques to assure a configurable storage system is described (SPL $\rightarrow$ Storage), e.g., SPL models used to model cloud applications~\cite{benlachgar2013review}.
Similar to the first perspective classification, we found publications considering both perspectives (4), e.g., using product-line techniques to configure robotics applications running on a configurable cloud~\cite{gherardi2014software}.

\subsection{Storage}
Second, we describe the results which are concerned with the data storage of a configurable system, including the type of storage, its organization structure, and the stored data.

\myPar{Type of Storage}
In the context of storage types, we found out that the authors usually mix the storage medium, e.g., a database, with its underlying infrastructure, e.g., an outsourced cloud system.
In 17 approaches, a database is implemented based on a cloud system. 
In 13 of these cases, it is explicitly stated that these are considered as SaaS or IaaS environments.
A combination of a cloud and a fog system with a database was described once~\cite{syed2016cloud}.
Overall, in 31 approaches a database was implemented where in 14 cases no underlying infrastructure is given.
In this context, five relational databases are described and in seven cases, SQL (usually MySQL) is given as database language.
Moreover, we identified eight papers where only a cloud system was generally given as storage system.

\myPar{Organization Structure}
We found all types of organization structures, i.e., centralized (44), distributed (14), and decentralized (1).
In one paper, no concrete structure could be extracted~\cite{siegmund2020dimensions}.
We note that it is not always clear whether the described or assigned organization is related to the storage medium (e.g., a database), the storage environment (e.g., a cloud), or the software system. 
There is usually insufficient information about the exact focus of the organizational structures in the papers
Cloud systems or service-oriented platforms (i.e., SaaS or IaaS) are usually referred as large-scale distributed (software) systems by definition~\cite{foster2008cloud,celesti2010security}.
However, this does not necessarily apply to the entire data storage system as such storage environments often consist of databases which are actual centralized.
We argue that only a few approaches are actual distributed in terms of the storage (i.e., regardless of the software infrastructure), e.g., an intercloud system based on resources and storage mediums distributed across multiple clouds~\cite{leite2016autonomic}.


\myPar{Stored Data}
We identified a variety of stored data.
However, three types of data seem most dominant: 17 times certain data (strongly) related to the variability of a system, including concrete features (5), feature and variability models (10), or variants (2), 14 times general application data (e.g., data of sensors~\cite{varela2020definition}), and 14 times source of the (configurable) software (e.g., parts of source code of a robotics system~\cite{gherardi2014software}). 
Other stored data include user data such as documents (5), application meta data (3), plugins (1), or software patches (1).
In five papers, the type of data was unspecified (4) or related to certain data models (1).
In addition, we assessed whether only the system or both system and user can access the stored data.
Overall, in 41 paper both system and user can access the data.
Accordingly, in 9 cases only the system can access the data.


\subsection{Security}
Third, the results regarding the security concerns of the configurable systems are described, including specification, standards, security goals, security threats, and proposed mitigation techniques.

\myPar{Standards}
We found five publications that mention or consider security standards or regulations in a certain way.
The ISO/IEC 27000 series, precisely ISO/IEC 27000~\cite{ISOIEC270002018} and ISO/IEC 27001~\cite{ISOIEC270012013}, are mentioned in two publications~\cite{fernandez2015patterns,garcia2015software}.
The NIST Cloud Computing Security Reference Architecture~\cite{nist2013nist} is also referenced by two papers~\cite{fernandez2015patterns,khan2017variability}.
Other standards are only addressed once and are quite diverse.
\citet{preuveneers2016systematic} referred to the secure payment standards PCI DSS and PA DSS~\cite{calderPCIDSS2011}. 
\citet{fernandez2015cloud} described SAML~\cite{hughes2005security} to securely exchange authentication and authorization data.
Moreover, \citet{shaaban2019ontology} referenced the industry communication-related standard IEC 62443~\cite{IEC62443} and IEEE 1686~\cite{IEEE1686} in the context of their configurable production system.
Only one publication~\cite{fernandez2015patterns} focus on the compliance with privacy regulations, precisely the HIPAA.

\myPar{Specification}
In most cases, security is considered as a general but important system goal (25),
Seven publications each mentioned security as a non-functional requirement, quality attribute, or a system requirement.
Four papers refer to security as a system feature.

\myPar{Goals}
Regarding the security goals described in the publication, most papers mention availability (38) or authorization (36) to certain degree, leading to a mix of a CIA triad and information security goal.
In 17 cases, accountability, often mentioned in the context of reliability, is addressed. 
Surprisingly, confidentiality and integrity, which are essential parts of the CIA triad, are only referred to 15 and 13 times.
Non-repudiation is addressed the least (9). 
Overall, we could not identify any papers that did not describe goals.
However, security goals are usually only mentioned and not described further, except integrity.
Summing up all mentions and concrete descriptions, security goals are only named 105 times in a certain context and described or explained in more detail in 57 times, which is almost a halving.

\myPar{Threats}
We extracted a variety of concerns that could threaten the security of the proposed systems.
Most papers (23) mention unauthorized data access and general privacy concerns (17).
In addition, the overall variability of configurable software systems and their storage (9) as well as the not sufficiently secured communication between clients, clients and software or storage, and software and storage (6) could lead to security issues.
Four papers mentioned trust in the overall configurable system as security threat, e.g., malicious administrators.
Other threats include software bugs and untrusted maintenance (3), data manipulation (2), data theft (2), general vulnerabilities (2), insecure hardware (1), and malware (1).

\myPar{Mitigation Techniques}
Mitigation techniques were usually given in a quite general way without details. 
Thus, most publications refer to encryption mechanisms or security protocols (26), e.g., symmetric encryption algorithms such as AES~\cite{varela2021carmen}, or network secrity protocols such as SSH~\cite{leite2016autonomic}, SSL~\cite{preuveneers2016systematic}, or TLS as successor technology to SSL~\cite{varela2020definition}.
In 18 papers, access controls mechanisms are proposed, e.g., account systems~\cite{mohamed2017integrated}.
Other mitigation techniques included data isolation (8), firewalls (4), signature (2), software misuse patterns (2), security measure models (1), and parallel variant execution (1).
Interestingly, in two papers by Krieter et al.~\cite{krieter2018towards,krieter2019using} the application of product-line techniques based on a detailed described security technology called Intel SGX~\cite{Costan2016IntelExplained} is given, fulfilling all security goals except accountability.

\subsection{Configurable System}

\myPar{Variability Focus}

\myPar{Projection}

\myPar{Verification}

\myPar{Evolution}

\myPar{Tool Support}

\bgroup
\rowcolors{0}{white}{gray!25}
\begin{table*}[t]
	\caption{\todo{ggf. 2 Tabellen draus machen}}
	\label{tab:security_main}
	\vspace*{-2ex}
	\centering \begin{adjustbox}{max width = 0.75\textwidth}

	\begin{tabular}{c*{28}{C{\fixedColWidth}}}
		\hiderowcolors
		& & & & & & & & & & & \multicolumn{6}{c}{Security goals} & & & & & & & & & \\\cmidrule[0.4pt](r{0.2em}l{0.1em}){12-17}%
		 &
		 &
		\multicolumn{4}{c}{\tabrotate{Perspectives}} &
		\multicolumn{4}{c}{\tabrotate{Storage}}
		&
		&
	    \multicolumn{3}{c}{\tabrotate{CIA triad}} &
		\multicolumn{3}{c}{\tabrotate{\parbox{1.7cm}{Information security}}} &
		&
		&
		&
		&
		\multicolumn{3}{c}{\tabrotate{Projection}} &
		\multicolumn{3}{c}{\tabrotate{Verification}} &
		&
		\\
		\cmidrule[0.4pt](r{0.2em}l{0.1em}){3-6}%
		\cmidrule[0.4pt](r{0.2em}l{0.1em}){7-10}%
		\cmidrule[0.4pt](r{0.2em}l{0.1em}){12-14}%
		\cmidrule[0.4pt](r{0.2em}l{0.1em}){15-17}%
		\cmidrule[0.4pt](r{0.2em}l{0.1em}){22-24}%
		\cmidrule[0.4pt](r{0.2em}l{0.1em}){25-27}%
		
		\tabrotate{Reference} & 
		\tabrotate{Domain} &
		\tabrotate{Security $\rightarrow$ SPL} & 
		\tabrotate{SPL $\rightarrow$ Security} & 
		\tabrotate{Storage $\rightarrow$ SPL} & 
		\tabrotate{SPL $\rightarrow$ Storage} & 
		\tabrotate{Type of storage} & 
		\tabrotate{Organization structure} & 
		\tabrotate{Stored data type} & 
		\tabrotate{Data access} & 
		\tabrotate{Standard} & 
		\tabrotate{Confidentiality} & 
		\tabrotate{Integrity} & 
		\tabrotate{Availability} & 
		\tabrotate{Authorization} & 
		\tabrotate{Accountability} & 
		\tabrotate{Non-repudiation} & 
		\tabrotate{Specification} &
		\tabrotate{Security threats} &
		\tabrotate{Security technology} &
		\tabrotate{Variability focus} &
		\tabrotate{Problem space} & 
		\tabrotate{Solution space} & 
		\tabrotate{Mapping} & 
		\tabrotate{Product-based} & 
		\tabrotate{Family-based} & 
		\tabrotate{Feature-based} & 
		\tabrotate{Evolution} &
		\tabrotate{Tool support}
		\\
		\toprule
		\showrowcolors
		\cite{benlachgar2013review} & & & & & & & & & & & & & & & & & & & & & & & & & & & & \\
		\cite{galster2013constraints} & & & & & & & & & & & & & & & & & & & & & & & & & & & & \\
		\cite{marinho2013mobiline} & & & & & & & & & & & & & & & & & & & & & & & & & & & & \\
		\cite{gherardi2014software} & & & & & & & & & & & & & & & & & & & & & & & & & & & & \\
		\cite{matar2014towards} & & & & & & & & & & & & & & & & & & & & & & & & & & & & \\
		\cite{moens2014feature} & & & & & & & & & & & & & & & & & & & & & & & & & & & & \\
		\cite{moens2014cost} & & & & & & & & & & & & & & & & & & & & & & & & & & & & \\
		\cite{nguyen2014exploring} & & & & & & & & & & & & & & & & & & & & & & & & & & & & \\
		\cite{parra2014soa} & & & & & & & & & & & & & & & & & & & & & & & & & & & & \\
		\cite{walraven2014efficient} & & & & & & & & & & & & & & & & & & & & & & & & & & & & \\
		\cite{azzolini2015evolving} & & & & & & & & & & & & & & & & & & & & & & & & & & & & \\
		\cite{baresi2015dynamically} & & & & & & & & & & & & & & & & & & & & & & & & & & & & \\
		\cite{brisaboa2015reusable} & & & & & & & & & & & & & & & & & & & & & & & & & & & & \\
		\cite{galindo2015supporting} & & & & & & & & & & & & & & & & & & & & & & & & & & & & \\
		\cite{fernandez2015patterns} & & & & & & & & & & & & & & & & & & & & & & & & & & & & \\
		\cite{fernandez2015cloud} & & & & & & & & & & & & & & & & & & & & & & & & & & & & \\
		\cite{fernandez2015patterns} & & & & & & & & & & & & & & & & & & & & & & & & & & & & \\
		\cite{garcia2015software} & & & & & & & & & & & & & & & & & & & & & & & & & & & & \\
		\cite{moens2015allocating} & & & & & & & & & & & & & & & & & & & & & & & & & & & & \\
		\cite{tizzei2015architecting} & & & & & & & & & & & & & & & & & & & & & & & & & & & & \\
		\cite{van2015variability} & & & & & & & & & & & & & & & & & & & & & & & & & & & & \\
		\cite{cao2015constraint} & & & & & & & & & & & & & & & & & & & & & & & & & & & & \\
		\cite{ali2016requirements} & & & & & & & & & & & & & & & & & & & & & & & & & & & & \\
		\cite{dig2016cope} & & & & & & & & & & & & & & & & & & & & & & & & & & & & \\
		\cite{jumagaliyev2016Evolving} & & & & & & & & & & & & & & & & & & & & & & & & & & & & \\
		\cite{leite2016autonomic} & & & & & & & & & & & & & & & & & & & & & & & & & & & & \\
		\cite{metzger2016coordinated} & & & & & & & & & & & & & & & & & & & & & & & & & & & & \\
		\cite{passos2016coevolution} & & & & & & & & & & & & & & & & & & & & & & & & & & & & \\
		\cite{perrouin2016complexity} & & & & & & & & & & & & & & & & & & & & & & & & & & & & \\
		\cite{preuveneers2016systematic} & & & & & & & & & & & & & & & & & & & & & & & & & & & & \\
		\cite{preuveneers2016feature} & & & & & & & & & & & & & & & & & & & & & & & & & & & & \\
		\cite{syed2016cloud} & & & & & & & & & & & & & & & & & & & & & & & & & & & & \\
		\cite{arrieta2015cyber}	& & & & & & & & & & & & & & & & & & & & & & & & & & & & \\
		\cite{alferez2017achieving} & & & & & & & & & & & & & & & & & & & & & & & & & & & & \\
		\cite{jalil2017adapting} & & & & & & & & & & & & & & & & & & & & & & & & & & & & \\
		\cite{leite2017dohko} & & & & & & & & & & & & & & & & & & & & & & & & & & & & \\
		\cite{mohamed2017integrated} & & & & & & & & & & & & & & & & & & & & & & & & & & & & \\
		\cite{munoz2017green} & & & & & & & & & & & & & & & & & & & & & & & & & & & & \\
		\cite{khan2017variability} & & & & & & & & & & & & & & & & & & & & & & & & & & & & \\
		\cite{butting2018controlled} & & & & & & & & & & & & & & & & & & & & & & & & & & & & \\
		\cite{krieter2018towards} & & & & & & & & & & & & & & & & & & & & & & & & & & & & \\
		\cite{aouzal2019policy} & & & & & & & & & & & & & & & & & & & & & & & & & & & & \\
		\cite{krieter2019using} & & & & & & & & & & & & & & & & & & & & & & & & & & & & \\
		\cite{shaaban2019ontology} & & & & & & & & & & & & & & & & & & & & & & & & & & & & \\
		\cite{varela2019cyberspl} & & & & & & & & & & & & & & & & & & & & & & & & & & & & \\
		\cite{assunccao2020variability} & & & & & & & & & & & & & & & & & & & & & & & & & & & & \\
		\cite{siegmund2020dimensions} & & & & & & & & & & & & & & & & & & & & & & & & & & & & \\
		\cite{varela2020definition} & & & & & & & & & & & & & & & & & & & & & & & & & & & & \\
		\cite{varela2021carmen} & & & & & & & & & & & & & & & & & & & & & & & & & & & & \\
		\cite{zhang2021evolutionary} & & & & & & & & & & & & & & & & & & & & & & & & & & & & \\
		\bottomrule
		\midrule[-2pt]			
	\end{tabular}
	\end{adjustbox}
\vspace*{-2ex}
\end{table*}
\egroup