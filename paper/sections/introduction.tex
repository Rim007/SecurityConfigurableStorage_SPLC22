\section{Introduction} \label{sec:introduction}

In recent years, the amount of data (e.g., images, audios, logs) massively increases due to the growing complexity, interconnection, and interaction of software systems~\citep{sagiroglu2013big,oussous2018big}.
To handle this data in an appropriate way, it is stored, managed, or even processed in a variety of different data storage systems, ranging from centralized databases to on-demand software-as-a-service (SaaS) cloud storage to decentralized blockchains~\citep{gabel2017secure,cusumano2010cloud,shafagh2017towards}.
A growing amount of these systems is configurable, i.e., these solutions are based on variability allowing their adaptation to specific requirements, e.g., customer demands, industry standards, hardware limitations, or legal regulations.
Moreover, the configurability allows the derivation of families of similar but customized variants by enabling or disabling certain features.
So, since each variant of a system includes individual functions and potential interactions between them, it contributes to the uniqueness but also the complexity of every solution~\citep{pohl2005software}. 

However, the data to be stored is usually private or security sensitive.
Consequently, to protect this data, it is essential to process and store it in a secure way by relying on concrete security models or standards (e.g., ISO/IEC 27000:2018~\citep{ISOIEC270002018})~\citep{Sartakov2018STANliteLevel,Popa2011CryptDB:Processing}. 
Due to the uniqueness and complexity of configurable storage systems as well as their underlying configurable software systems based on a variety of variants, there is a particularly high risk for both configurable systems of becoming the subject of cyber attacks~\citep{Kenner2021MappingStudy}.
These attacks are often based on the exploitation of certain system vulnerabilities or malicious administrators, leading to unauthorized data access, data loss or even data manipulation~\cite{Gunawi2014WhatCloud, Campanile2017CloudifyingDomain}.

\todo{Ist die Einteilung in configurable software systems und configurable storage systems sinnvoll (mit Oberbegriff configurable systems)?}

Consequently, research on configurable systems, particularly those based on software product lines (SPL), and security has been conducted in recent years.
This research include both approaches for configurable software systems and configurable storage systems.
For instance 

\todo{Beispiele}


Although there exist extensive research on security and configurable systems, especially involving storage systems, we are missing an overview of current state-of-the-art research.
Consequently, there is no comprehensive systematization of configurable storage systems, their security and variability concerns, or future research directions.
By conducting a systematic mapping study, we aim to addressed this research gap~\cite{Kitchenham2015EvidenceSLR}.
Precisely, we analyzed 53 papers proposing configurable storage systems with a consideration of security concerns that are published on \textsc{Scopus}\footnote{www.scopus.com}, \textsc{IEEE Xplore}\footnote{ieeexplore.ieee.org}, and the \textsc{ACM Guide to Computing Literature}\footnote{dl.acm.org/search/advanced} within the last decade (2013--2022).
This way, we are able to summarize the current state-of-art research and emphasize research gaps as well as relevant research direction regarding the intersection of security and SPL-related configurable storage systems.

Overall, we contribute the following with our study:
\begin{itemize}[nosep]
	\item A systematic and comprehensive overview of recent research on security in the context of configurable storage systems.
	
	\item A discussion on what properties are covered and for which properties further investigations is required.
	
	\item An open-access repository including the detailed study results to increase the study's overall comprehensibility and to ensure a replicability of our study.\footnote{\url{LINK}\label{ft:repo}}.
\end{itemize}


We argue that our study results and discussion findings can help both researchers and practitioners in understanding the recent state-of-the-art as well as open research directions.

\todo{problem: oft data storage nicht allein, sondern in kombination mit configurable software system beschrieben, teils storage selbst nicht konfigurierbar, sondern nur fürs Speichern von variants genutzt. Wie gehen wir damit um?}