\section{Related Work} \label{sec:relatedwork}

We found six related papers contributing literature reviews or mapping studies in the field of configurable systems also involving security or security.
However, security is usually only extracted or mentioned as one of several quality attributes or non-functional requirements from a high-level perspective, without a detailed analysis of security concerns (e.g., threats, attack mitigation techniques).
Moreover, data storages are typically not addressed in the context of configurable systems and security.
In contrast to the related work we found, our study provides a comprehensive and systematic overview of security in the context of configurable data storages, providing a detailed analysis of these aspects.  

\looseness=-1
\citet{myllarniemi2012systematically} conducted a literature review of 29 papers (2000--2010) focusing on the variability of quality attributes that are part of SPLs.
The authors considered security as a quality attribute from a high-level perspective.  
\citet{benlachgar2013review} reviewed four SPL models for SaaS applications (without any time restriction), including an assessment of the models' relevance.
However, they do not consider security aspects. 
\citet{mahdavi-hezavehiVariabilityQualityAttributes2013} report a literature review of 46 papers (2000--2011) focusing on the variability of quality attributes of service-based software systems, including general security goals.
The authors found that only a few papers actually consider security in a general way as quality attributes.
\citet{hammaniSurveyNonFunctionalRequirements2014b} surveyed non-functional requirements in nine papers in the context of modeling and verifying SPLs (without any time restriction).
However, they only provide a high-level overview regarding security without focusing on details of the security concerns or data storages.
\citet{geraldi2020software} reviewed 56 SPL-related papers (2006--2018) that describe concepts or applications in the context of the internet-of-things. 
As cloud systems are closely associated with the internet-of-things, they are considered as part of the study, but not examined in detail. 
Security is only generally considered from a high level in the context of non-functional requirements.
\citet{Kenner2021MappingStudy} presented a systematic mapping study of 65 papers (2011--2020) with a focus on safety and security for configurable software systems, which is the study most closely related to our own.
We complement this mapping study by providing a detailed analysis of security research on configurable data storages, a perspective not scratched in the study of \citeauthor{Kenner2021MappingStudy}.