\section{Background} \label{sec:background}

In the following section, we provide relevant background information on \emph{configurable systems}, \emph{data storage}, and \emph{security}.


\subsection{Configurable Systems}\label{sec:background:configsystems}

\subsection{Data Storage}\label{sec:background:storage}

Data storage, i.e., a solution that is able to store specific data in a certain technological way, is usually generally classified by its organization structure, i.e., centralized or decentralized~\cite{furht2010CloudFundamentals, chowdhury2018BlockchainAnalysis}. 
Centralized storage solutions, e.g., relational SQL databases, are built around one single point (i.e., a server) handling all major processing or storage task.
Consequently, all clients are connected to the central point where the data is stored~\cite{ramamritham1996RealTimeDatabases,chowdhury2018BlockchainAnalysis}.
In contrast, decentralized storage solutions, e.g., decentralized NOSQL databases or blockchain applications, rely on the distribution of processing or storage steps among several machines.
Thus, the dependency on individual processing points is much weaker than in centralized solutions.
In addition, it guarantees a greater scalability than centralized solutions\cite{ayoade2018DecentralizedEnvironment,mcconaghy2016BigchainDB:Database,Bao2020WhenIssues}.

Data storage can be realized based on self-hosting or outsourcing.
In the context of outsourcing, which has become increasingly widespread for enterprises in recent years, there are several common technologies, especially cloud, edge, and fog~\cite{gabel2017SecureExecution,correia2020SafeguardingEdge}.
These environments are created based on a complex combination of software and hardware components able to provide server-based storage space at a high level of efficiency, flexibility, scalability, and on-demand data access based on the Internet of Things~\cite{markovic2013SmartComputing,silva2018SecurityComputing}.  
Outsourcing solutions usually serve as the technology where storage and processing solutions can be built on, leading to the creation of service-based storage and software, usually called Software as a service (SaaS)~\cite{lee2009quality,kulkarni2012cloud}.


\subsection{Security}\label{sec:background:security}

