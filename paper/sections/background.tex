\section{Background} \label{sec:background}

In the following section, we provide relevant background information on \emph{configurable systems}, \emph{data storage}, and \emph{security}.


\subsection{Configurable Systems}\label{sec:background:configsystems}

A configurable system is characterized by a number of  (diverse) features (i.e., user-visible functions) that can be customized to meet customer needs, e.g., specific customer or user requirements, hardware constraints, or even legal regulations.


\subsection{Data Storage}\label{sec:background:storage}

Data storage, i.e., a medium that is able to store specific data in a certain technological way, is usually generally classified by one of three main organization structures, i.e., centralized, decentralized, or distributed~\cite{furht2010CloudFundamentals,wu2017distributed}. 
Centralized storage solutions, e.g., centralized relational SQL databases, are built around one single point (i.e., a server) handling all major processing or storage task.
Consequently, all machines are connected to the central point where the data is stored~\cite{ramamritham1996RealTimeDatabases,chowdhury2018BlockchainAnalysis}.
In contrast, decentralized storage solutions, e.g., decentralized NOSQL databases in the context of blockchain applications, rely on the distribution of processing or storage steps among several machines with no or limited coordination~\cite{wu2017distributed,ayoade2018DecentralizedEnvironment}.
Thus, the dependency on individual processing points is much weaker than in centralized solutions~\cite{ayoade2018DecentralizedEnvironment,mcconaghy2016BigchainDB:Database,Bao2020WhenIssues}.
If a decentralized system provides a (close) coordination between independent machines, it is called a distributed system~\cite{van2002distributed,wu2017distributed}.
For instance, intercloud environments relying on a variety of storage mediums are usually associated with this organization structure~\cite{celesti2010security,leite2016autonomic}. 

Data storage mediums usually rely on certain storage environments that are oriented towards self-hosting (i.e., an environment located on a local machine) or outsourcing.
In the context of outsourcing, which has become increasingly widespread for enterprises in recent years, there are several common technologies, especially cloud, edge, and fog~\cite{gabel2017secure,correia2020SafeguardingEdge}.
These environments are created based on a complex combination of software and hardware components able to provide server-based storage space at a high level of efficiency, flexibility, scalability, and on-demand data access based on the Internet of Things~\cite{markovic2013SmartComputing,silva2018SecurityComputing}.  
Outsourcing solutions often serve as the technology where data storage and data processing can be built on, leading to the creation of service-based systems or infrastructures, usually called Software as a Service (SaaS) or Infrastructure as a Service (IaaS)~\cite{lee2009quality,kulkarni2012cloud,serrano2015infrastructure}.


\subsection{Security}\label{sec:background:security}

To consider security from a practice-oriented perspective, it is essential to rely on established as well as pertinent norms and standards, e.g., the ISO/IEC 27000 series~\cite{ISOIEC270002018}, ISO/IEC 25010~\cite{ISOIEC250102011}, ISO/IEC 29100~\cite{ISOIEC291002011}, or the NIST Guide for Conducting Risk Assessments~\cite{guidegonductingrisk2012}.
According to these established standards, security is a characteristic of a software system aimed to protect its stored or processed data or information against a wide range of threats regarding caused by attacks, vulnerabilities, errors, or even nature (e.g., in the context of safety-critical systems)~\cite{ISOIEC250102011,ISOIEC270002018}.
In this context, a threat is defined as an unwanted, but possible event or incident, resulting in a harm of a system.
If there is a concrete possibility of exploiting such a threat, e.g. in terms of vulnerabilities, this condition is referred to as a security risk~\cite{ISOIEC270002018,guidegonductingrisk2012}.
To minimize such risks as much as possible, the regular assessment of them in terms of defined monitoring, measurement, and analysis processes according to the data lifecycle of the individual system is essential~\cite{ISOIEC270012013,ISOIEC270042016}.
Furthermore, risks are also minimized by fulfilling defined security goals as relevant security subcharacteristics~\cite{ISOIEC270002018,guidegonductingrisk2012}.
According to the ISO/IEC 27000 series~\cite{ISOIEC270002018} and ISO/IEC 25010~\cite{ISOIEC250102011} these typically include three goals which are known as the CIA triad~\cite{lundgren2019defining,samonasCIAStrikesBack2014}:

\begin{description}
    \item\textbf{Confidentiality:} The data is only available for authorized entities.
	\item\textbf{Integrity:} The data can only be modified by an authorized entity.
	\item\textbf{Availability:} The system ensures timely and reliable access to its data for authorized entities.
\end{description}

\noindent In the context of information security, these are usually extended by accountability, authenticity, and non-repudiation~\cite{ISOIEC250102011,ISOIEC270002018,guidegonductingrisk2012}.

\begin{description}
    \item\textbf{Accountability:} Any action can be traced to a unique entity.
	\item\textbf{Authenticity:} The identity of a person can be clearly proven to be the one claimed.
	\item\textbf{Non-repudiation:} It is possible to prove the occurrence of every action and what entities were involved.
\end{description}

\noindent Information security is also often associated with further goals, e.g. authentication, reliability, or privacy~\cite{ISOIEC270002018,ISOIEC291002011}.
However, these are usually subordinated to the six goals we described, or their fulfillment is significantly supported by the fulfillment of the six goals, e.g., privacy of personal data.

		