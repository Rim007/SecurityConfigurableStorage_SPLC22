\section{Background} \label{sec:background}

In the following section, we provide relevant background information on \emph{configurable systems}, \emph{data storages}, and \emph{security}.


\subsection{Configurable Systems}\label{sec:background:configsystems}

A configurable system is characterized by a number of features (i.e., user-visible functionalities~\cite{Apel2013FOSPL}) that can be configured to meet customer needs, such as user requirements, hardware constraints, or legal regulations.
We classify configurable systems into two categories: configurable software systems (e.g., plugin-based systems) and configurable storage systems (e.g., cloud-based systems).
Usually, such systems rely on concepts, methods, and techniques related to SPLs~\cite{Apel2013FOSPL,pohl2005software}.
Configurable features in an SPL are managed through variability mechanisms (e.g., variability models, such as feature models) to organize, implement, and document features with their dependencies~\cite{Kang1990FODA,Apel2013FOSPL,Czarnecki2012CoolFeatures,Schaefer2012SoftwareDiversity,Nesic2019Principles}.
Such mechanisms are typically supported by tools that check whether a configuration is valid, propagate configuration decisions~\cite{Krieter2016ModelSlicing,Apel2013FOSPL}, and derive valid variants automatically (e.g., FeatureIDE~\cite{Meinicke2017FeatureIDE}, pure::variants~\cite{Beuche2012purevariants}).
Consequently, configurable systems can be defined using the definitions of problem space (i.e., a domain abstraction), solution space (i.e., the implementation), and a mapping between both spaces (i.e., connection between both spaces allowing to derive variants automatically)~\cite{Apel2013FOSPL}.
Moreover, every configurable system can be verified based on certain attributes.
For this purpose, three established strategies exist: feature-based (i.e., analyzing each feature in isolation without considering configurations or dependencies), product-based (i.e., analyzing a system configuration based on its code or an abstraction), and family-based (i.e., analyzing the whole configurable system including valid configurations)~\cite{Thuem2014AnalysisSurvey}.


\subsection{Data Storages}\label{sec:background:storage}

A data storage, a medium that is able to store specific data in a certain way, is usually classified according to one of three structures it employs, namely centralized, decentralized, or distributed~\cite{furht2010CloudFundamentals,wu2017distributed}. 
Centralized storages (e.g., centralized relational SQL databases) are built around one single unit handling all major processing or storage tasks (e.g., one server).
Consequently, all machines are connected to the central unit where the data is stored~\cite{ramamritham1996RealTimeDatabases,chowdhury2018BlockchainAnalysis}.
In contrast, decentralized storages (e.g., decentralized NOSQL databases in the context of blockchain applications) rely on the distribution of processing or storage steps among several units with no or limited coordination~\cite{wu2017distributed,ayoade2018DecentralizedEnvironment}.
So, the dependency on an individual processing unit is much weaker than in centralized solutions~\cite{ayoade2018DecentralizedEnvironment,mcconaghy2016BigchainDB:Database,Bao2020WhenIssues}.
If a decentralized storage provides (close) coordination between independent units, it is called a distributed storage~\cite{van2002distributed,wu2017distributed}.
For instance, inter-cloud environments relying on a variety of storages are usually associated with this structure~\cite{celesti2010security,leite2016autonomic}. 

Data storages usually operate in certain environments that are oriented towards self-hosting (i.e., an environment located on a local machine) or outsourcing.
In the context of outsourcing, which has become increasingly widespread for enterprises in recent years, there are several common technologies, such as cloud, edge, and fog computing~\cite{gabel2017secure,correia2020SafeguardingEdge}.
Such environments use a complex combination of software and hardware components, and are able to provide server-based storage space at a high level of efficiency, flexibility, scalability, and on-demand availability~\cite{markovic2013SmartComputing,silva2018SecurityComputing}.  
Outsourcing solutions often serve as an underlying technology based on which data storing and processing functionalities can be implemented, leading to service-based systems or infrastructures; usually called SaaS or Infrastructure as a Service (IaaS)~\cite{lee2009quality,kulkarni2012cloud,serrano2015infrastructure}.


\subsection{Security}\label{sec:background:security}

According to norms and standards established in practice (e.g., ISO/IEC 27000 series~\cite{ISOIEC270002018}, ISO/IEC 25010~\cite{ISOIEC250102011}, ISO/IEC 29100~\cite{ISOIEC291002011}, NIST Guide for Conducting Risk Assessments~\cite{guidegonductingrisk2012}), security is a property of a software system aimed at protecting stored or processed data against a wide range of threats caused by attacks, vulnerabilities, errors (i.e., bugs), or nature (e.g., in the context of hardware).
A threat is defined as an unwanted, but possible, event that results in a harm to a system.
If there is a concrete possibility of exploiting such a threat, for instance, in terms of vulnerabilities, the threat poses a security risk~\cite{ISOIEC270002018,guidegonductingrisk2012}.
To minimize such risks, a regular risk assessment using defined monitoring, measurement, and analysis processes according to the data lifecycle of the individual system is essential~\cite{ISOIEC270012013,ISOIEC270042016}.
Furthermore, risks are also reduced by fulfilling defined security goals implemented via mitigation techniques, such as cryptographic control mechanisms~\cite{ISOIEC270002018,ISOIEC270012013,guidegonductingrisk2012}.
According to the ISO/IEC 27000 series~\cite{ISOIEC270002018} and ISO/IEC 25010~\cite{ISOIEC250102011}, three security goals are particularly important, which are also known as the CIA triad~\cite{lundgren2019defining,samonasCIAStrikesBack2014}:
\begin{itemize}[nosep,leftmargin=*]
	\item\emph{Confidentiality:} The data is only available to authorized entities.
	
	\item\emph{Integrity:} The data can only be modified by an authorized entity.
	
	\item\emph{Availability:} The system ensures timely and reliable access to its data for authorized entities.
\end{itemize}
In the context of information security, these three goals are usually extended by~\cite{ISOIEC250102011,ISOIEC270002018,guidegonductingrisk2012}:
\begin{itemize}[nosep,leftmargin=*]
	\item\emph{Accountability:} Any action can be traced to a unique entity.
    
	\item\emph{Authenticity:} The identity of an entity can be clearly proven to be the one claimed.
	
	\item\emph{Non-repudiation:} It is possible to prove the occurrence of every action and what entities were involved.
\end{itemize}
Information security is often associated with further security goals, such as reliability or privacy~\cite{ISOIEC270002018,ISOIEC291002011}.
However, such goals are usually subordinated to the six we described, have particular legal dependencies (e.g., legal regulations such as GDPR or HIPAA~\cite{tovino2017hipaa}), or their fulfillment is significantly related to or supported by the fulfillment of the six goals.

		