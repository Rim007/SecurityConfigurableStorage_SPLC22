\section{Discussion} \label{sec:discussion}

After describing the extraction results, we give an overview of relevant core insights derived from the results.
Precisely, we present XY \textbf{literature gaps (LG)} we found in the analyzed publications after collaboratively analyzing and discussing the results among this study's authors.
The overall discussion section is oriented towards the three topics of this paper, namely storage, security, and configurable systems. 
Moreover, these three topics are discussed in concert.

\myPar{Storage}
Interestingly, we found out that the concrete understanding of what storage actually is, varies heavily between the selected publications.
In detail, storage is either referred to as concrete medium (e.g., a database), the environment of the medium (e.g., a configurable cloud system), or the cooperation of both, usually with a focus on the environment. 
Our understanding refers to storage as a medium embedded into an environment, which has actually no concrete storing abilities without implementing the medium.
However, we assume that storage is usually used as a generic term for systems which are concerned with certain storage goals or functionalities, leading to \textbf{LG\textsubscript{1}}: \emph{A uniform definition and understanding of storage, especially in the area of configurable systems, is needed} in order to ensure comparable analyses and considerations of storage and the stored data (e.g., in the context of security).

Surprisingly, the uniform assignment of the storage organization structure (i.e., centralized, decentralized, distributed) was hardly possible since the concrete perspective of the publications was not clear.
Precisely, we assume that there exist four possible layers (i.e., perspectives) regarding the organization structure of a system with storage abilities:
\begin{enumerate}[nosep]
	\item the storage medium, e.g., a database,
	\item the storage environment, e.g., a cloud system, 
	\item the software system, e.g. a medical system, and
	\item storage hardware, e.g., the server infrastructure.
\end{enumerate}
Each layer might contain systems that are either centralized, decentralized or distributed.
Since such (\textbf{LG\textsubscript{2}}) \emph{uniform definitions regarding the perspective of the organization structure} were not given in any of the papers, comparable assignments are not possible at the moment.
However, we assume that the organizational structure described in the publications usually refers to the storage environment in combination with the software system (e.g., distributed, multi-tenant systems).

Our results also highlight that usually only the software system or software system and storage are actually variable. 
This finding shows the great dependency and interactions of storage systems and software systems in the context of variability, i.e., both should be considered together if possible.
In addition, the actual configurability of the storage mediums, at least the storage environments which are configurable and scalable by definition, is rarely addressed.
This may be due to an insufficient relevance of these systems in the field of product-line engineering or due to a lack of concrete solutions and research in this area.
Nevertheless, we argue that \textbf{(LG\textsubscript{3})} \emph{more research is needed regarding the variability of storage mediums and storage environment taking into account their interactions with configurable software systems}.

\myPar{Security}
Based on the individual fulfillment of all security-related criteria, we state the \textbf{(LG\textsubscript{4})} \emph{general need for security-related considerations for configurable software systems and their storage}.
This need is particularly relevant in the context of addressed norms, standards, or legal regulations. 
Unsurprisingly, usually no security standard is addressed by the publications, although there exist a variety of international standards published by the NIST, ISO, or IEC. 
In only two cases, the most relevant standards of the ISO/IEC 27000 series~\cite{ISOIEC270002018} are described which provide essential definitions (ISO/IEC 27000~\cite{ISOIEC270002018}), requirements (ISO/IEC 27001~\cite{ISOIEC270012013}), or ways of monitoring, measuring, analyzing, and evaluating security concerns (ISO/IEC 27004~\cite{ISOIEC270042016}).
Moreover, relevant legal regulations (usually in the context of privacy for personal (medical) data) are only given in one paper.
Thus, we state that \textbf{(LG\textsubscript{5})} \emph{security for configurable software systems and their storage systems is lacking practice orientation}, leading to a major hurdle in the context of the transfer of theoretical knowledge into business practice.
Possible reasons for not referring to established standards could be the missing need for a practical standard in the context of purely theoretical considerations or the assumption that the configurable applications automatically rely on ISO/IEC 27000 series as it is one of the most established security standards.
Even the limited distribution of certain systems and variability mechanisms in practice (e.g., configurable robotics cloud applications in Industry 4.0~\cite{gherardi2014software}) may result in a insufficient need for protecting these systems against potential cyber attacks based on established standards.

Our findings also emphasize that \textbf{(LG\textsubscript{6})} \emph{there is a lack of concrete specifications of mitigation techniques} (e.g., concrete encryption algorithms such as AES-256 or RSA-2048).
For instance, in the context of encryption it is not clear what is encrypted (e.g., only storage medium or only its communication) and how it is encrypted (e.g., using 1024 or 2048 Bit in the context of asymmetric RSA encryption).
Such measures are often described in a general way (e.g., firewalls~\cite{syed2016cloud} or access control~\cite{moens2015allocating}) resulting in no or only limited practical relevance in the context of actual cyber attacks (e.g., man-in-the-middle or brute force attacks). 
Especially in the case of configurable systems, i.e. software systems, their storage systems, and the communication between both, a specification that is as precise as possible is essential because of the increased attack surface due to variability~\cite{Kenner2021MappingStudy}.
However, we suggest not only specifying mitigation techniques, but also assigning them to concrete threats or risks (i.e., cyber attacks) to increase the practical relevance.

Similar to data storage, the perspective of the security technique implementations differs in the analyzed papers.
In detail, some measures refer to the software system (e.g., firewall) while others focus on the storage environment (e.g., access control) or storage medium (e.g., encryption).


differentiated perspectives on security is relevant, usually no focused consideration of individual component, e.g., database. application often rely on security measure for overall system, e.g., firewall, IDS IPS...
no detailed description of concrete encryption,

However, we argue that \textbf{(LG\textsubscript{6})} \emph{security (including privacy) is lagging behind the technological advances regarding configurable systems and their data storage}, which is a know problem in the context of data storage~\cite{strohbach2016big}.



verschiedene Schutzbereiche, je nachdem welche Daten geschützt werden (bspw. persönliche Daten oder medizinische Daten, verschiedene Prioritäten von Daten (z.B. sensoren in Industry 4.0))

Availability und Authorization zusammenhang zu Datenbank, klassische Datenbank ziele, unbewusst richtig gemacht auch ohne Bezug zu Security/CIA
manche security goals angenommen, weil DB das mit sich bringt (Integrity, Non-repudiation)

Unterscheidung von Authorization, Authentification --> ist aber genormt!

\myPar{Configurable Systems}
domains diverse meaning that its usually fundamental research which is not domain specific, more research in collaboration with practitioners is needed
viel implementierung (solution space) 


\myPar{Configurable Systems, Data Storage and Security in Concert}

Moreover, we identified that the databases described in the publications are usually not explained in detail, e.g. whether they are SQL or NoSQL databases or how variable they actually are.
We argue that every type or even variant of a database can possibly lead to new requirements which are relevant for systems, models, or patterns interacting with them, e.g., the underlying software system or concrete security measures such against SQL injection attacks~\cite{humayun2020cyber}.
In addition, it is not clear what impact the general type of databases (e.g., relational), their variability (e.g., in terms of version updates), or the automation of certain database back-end processes (e.g., SQLite for mobile devices) have regarding potential security risks.
This is why we argue that \textbf{(LG\textsubscript{2})} \emph{more research is needed in the context of variable storage mediums, i.e., databases, to understand relevant relationships and to minimize security risks cause by storage variability}.

ggf. security as a feature, um security nicht mehr zu vergessen (z.B. modellierung in FeatureIDE) --> awareness, Sicherheit als konfigurierbar betrachten, wie high-level wird security angesetzt und betrachtet


\subsection{Threats to Validity}

We are aware of some threats that could impair both internal and external validity of our mapping study.
First, the papers' authors usually do not share the same understanding of certain terms and definitions.
This threat is particularly relevant in the context of what is actually understood as storage, e.g., a database as a storage medium or a cloud system as a storage environment.
Second, there is  a lack of completeness in depth of content and consistency in the context of the analyzed papers.
In detail, some authors described their work in great detail, while others only mentioned actually essential properties (e.g., concrete encryption algorithms) or described them only briefly.
However, this is probably due to the fact that we considered both conference papers and journal articles in our study.
Although we aimed to mitigate potential issues in our data analysis, we cannot ensure that this did not impair the comparability or led to misinterpretations on our side.
Third, we found a few issues regarding the fulfillment of certain criteria, e.g., several publications did not provide any description or even a name of tools supporting their work leading to a decreasing comprehensibility and replicability of their work.
Fourth, besides these threats to the internal validity, the external validity could be impaired by the number of included publications (50) in our study.
Although we already considered three databases (i.e., \textsc{Scopus}, \textsc{IEEE Xplore}, and the \textsc{ACM Guide to Computing Literature}), we are aware of the fact that the smaller the number of included literature, the higher the potential of the impact of misclassifications.
In this context, we assume that we potentially missed papers that are related to the covered topics (i.e., security, configurable systems, data storage) due to our overall search strategy, e.g., the search string that does not cover all potential subtopics such as variability modeling.

Although the described issues may threaten our study findings, we aimed to mitigate them by only relying on established literature databases covering relevant and peer-reviewed publications.
Moreover, we considered a large number of papers in our overall systematical evaluation process (i.e., initially 538 publications fetched based on our search string), leading to a decreased threat of missing publications in this research area.
Thus, we argue that our study provides detailed findings regarding security in the context of configurable systems and story which are valuable for the research community.
For future work, we plan to extend our study by conducting a systematic literature review, e.g. examining the relevant topics and their criteria in more detail based on concrete metrics and proving the identified research trends and gaps. 
