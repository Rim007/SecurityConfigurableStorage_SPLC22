\section{Discussion} \label{sec:discussion}

After describing the data we extracted, we now discuss the core insights we obtained from our analysis.
For this purpose, we provide an overview of our synthesis and describe open \textbf{research opportunities (ROs)} that we derived from the results through collaborative analyses and discussions among this paper's authors.
We argue that these objectives require more intensive research, since they are not sufficiently described in our sample of papers---which we consider representative of our research community, seeing the diverse set of researchers involved as well as domains and research reported.


\subsection{Configurable Storages}

Interestingly, we found that the concrete understanding of what a data storage actually is varies heavily across the selected papers.
Precisely, data storage is either referred to as a medium (e.g., a database), the environment of the medium (e.g., a configurable cloud system), or the cooperation of both; usually with a focus on the environment. 
Our understanding refers to data storage as a medium embedded into an environment, which has actually no concrete storing abilities without implementing the medium.
However, we assume that data storage is usually used as a generic term for systems that are concerned with certain storage goals or functionalities, leading to our argument \textbf{(RO\textsubscript{1})}: \emph{A uniform definition and understanding of data storages, especially in the area of configurable systems, is needed} to ensure comparable analyses and research on data storages and the data stored (e.g., in the context of security).

Surprisingly, a uniform assignment of each storage's structural organization (i.e., centralized, decentralized, distributed) was hardly possible, since the concrete perspective of a paper was often unclear.
Precisely, we assume that there exist several possible layers (i.e., perspectives) regarding the structure of a system with a data storage, for instance:
\begin{enumerate}[nosep,leftmargin=*]
	\item the storage medium (e.g., a database),
	\item the storage environment (e.g., a cloud system), 
	\item the software system (e.g. a medical system), and
	\item the storage hardware (e.g., the server system infrastructure).
\end{enumerate}
Each layer might contain systems that are either centralized, decentralized, or distributed.
Since \textbf{(RO\textsubscript{2})} \emph{uniform definitions regarding the perspectives of the structural organization are not defined} in any of the papers, comparable assignments are not or hardly possible at the moment.
However, we assume that the structural organization described in the papers usually refers to the storage environment in combination with the software system (e.g., distributed, multi-tenant cloud systems), where the actual storage medium is usually centralized.
We strongly recommend to initiate further research in the context of the structural organization of configurable data storages, for instance, defining and formalizing the different perspectives in terms of a framework.

Our results further highlight that usually only the software system, or both the software system and data storage combined, are actually variable. 
This insight indicates the dependency and interaction of storages and software systems in the context of variability, namely that both should be considered together if possible.
Interestingly, we could not identify research concerned with how variability of the system impacts the data in the storage (e.g., configuring the data itself).
In addition, the actual configurability of the storage medium, at least the storage environment that is configurable and scalable by definition (e.g., clouds), is rarely addressed.
This may be due to an insufficient relevance of these systems in the field of SPLs or due to a lack of concrete solutions and research in this area.
Nevertheless, we argue that \textbf{(RO\textsubscript{3})} \emph{more research is needed regarding the variability of storage mediums and environments, taking into account their interactions with configurable software systems}.

\subsection{Security}

%WHAT???
%Based on the individual fulfillment of all security-related criteria, there is a general need for security-related considerations for configurable software systems and their storage as research in security (including privacy) is lagging behind the technological advances regarding configurable systems and their data storage.

To achieve the security goals we defined, security-related concerns must be studied more extensively in the context of configurable storages, but this research is lagging behind the technological advances.
This is a well-known problem in the context of data storages~\cite{strohbach2016big}.
Particularly, this need arises with respect to norms, standards, and legal regulations. 
Unsurprisingly, there is usually no security standard addressed  in the papers, although there exists a variety of international standards published by NIST, ISO, or IEC. 
Only in two papers, the most relevant standards of the ISO/IEC 27000 series~\cite{ISOIEC270002018} are described to provide essential definitions (ISO/IEC 27000~\cite{ISOIEC270002018}), requirements (ISO/IEC 27001~\cite{ISOIEC270012013}), or ways of monitoring, measuring, analyzing, and evaluating security concerns (ISO/IEC 27004~\cite{ISOIEC270042016}).
Moreover, relevant legal regulations (usually in the context of privacy for personal and medical data) are only described in one paper.
So, we argue that \textbf{(RO\textsubscript{4})} \emph{security for configurable data storages is lacking practice orientation based on established norms, standards, and legal regulations}, leading to a major barrier regarding the transfer of theoretical knowledge into practice.
Possible reasons for not referring to established standards could be the missing need for a practical standard in the context of purely theoretical considerations or the assumption that the configurable systems automatically rely on the ISO/IEC 27000 series; since it is one of the most established security standards.
Also, the limited number of certain systems and variability mechanisms in practice (e.g., configurable robotics cloud applications in Industry 4.0~\cite{gherardi2014software}) may result in an insufficient interesting of researching how to protect these systems against potential cyber attacks based on established standards.

Our findings further emphasize that \textbf{(RO\textsubscript{5})} \emph{there is a lack of concrete specifications of mitigation techniques} (e.g., concrete encryption algorithms, such as AES-256 or RSA-2048).
For instance, in the context of encryption, it is not clear what is encrypted (e.g., data, storage medium, or communication) and how it is encrypted (e.g., using 1024 or 2048 bit in the context of asymmetric RSA encryption).
We argue that we need definitions of different protection levels and assigned security measures, depending on which data, medium, environment, or system (including variants) need to be protected.
For example, there is a different need for protecting personal data, such as the birth date or medical data of a person.
However, in the analyzed papers, security measures are often described in a general way (e.g., firewalls~\cite{syed2016cloud} or access control~\cite{moens2015allocating}) resulting in no or only limited practical relevance in the context of actual cyber attacks (e.g., man-in-the-middle or brute force attacks). 
Especially in the case of configurable systems, their data storages, and the communication between both, a specification that is as precise as possible is essential, due to the increased attack surface caused by the variability~\cite{Kenner2021MappingStudy}.
We suggest to not only specify mitigation techniques, but to also assign them to concrete threats or risks (i.e., cyber attacks) to increase the practical relevance of research.

Interestingly, similar to the data storages, the perspectives on the security measures differ in the analyzed papers.
In detail, some measures refer to the overall software system (e.g., firewall~\cite{siegmund2020dimensions}), while others focus on the storage environment (e.g., access control of cloud users~\cite{moens2015allocating}) or the storage medium (e.g., encryption of a database and its communication~\cite{preuveneers2016systematic}).
However, it is not always clear what exactly a security measure refers to.
Thus, we argue that \textbf{(RO\textsubscript{6})} \emph{a concrete assignment of mitigation techniques to their target area (i.e., their perspective) is usually missing}, leading to a decreasing comparability of existing research.

In contrast to other research on security and configurable systems~\cite{Kenner2021MappingStudy}, the security goals of the CIA triad (confidentiality, integrity, availability) are not addressed the most (cf. \autoref{fig:taxonomy}) within the papers we selected.
Instead, availability and authorization are typically mentioned.
We argue that this is the result of these two security goals being more closely related to data storages.
Furthermore, we assume that some security goals that are relevant, but have not been mentioned, are considered as given (e.g. integrity and non-repudiation).
In the context of databases, both can be assumed to be automatically fulfilled by the general properties of a database. 
However, we emphasize that \textbf{(RO\textsubscript{7})} \emph{concrete security goals (e.g., according to the definitions of ISO/IEC 27000~\cite{ISOIEC270002018}) are usually missing in the context of configurable storages}, implying a lack of practical orientation.

\subsection{Configurable Systems}

We identified a variety of domains, implying that most the of selected papers are not concerned with domain specifics. 
Thus, most of the analyzed research should be transferable and applicable to a large number of use cases.
However, we note that this might be a hurdle for actually implementing the corresponding techniques in practice, since these solutions are not yet specified for a concrete domain or use case. 
We identified a trend towards the production domain, more specifically cyber-physical systems and robotics applications.
So, we see a high application potential for security research on configurable data storages in cyber-physical, cloud, robotics, and Industry 4.0 systems---which share similar properties, involve large amounts of data, and have high security concerns.
Similarly, considering the projections used in the papers, most focus on the mapping between problem space and solution space or the problem space only, meaning that actual (implemented) solutions for secure configurable storages are missing.
Consequently, we argue that there is a need to \textbf{(RO\textsubscript{8})} \emph{provide sufficiently concrete solutions that can serve as working examples for practice in concrete domains and improve collaborations with practitioners}. 
Furthermore, we could hardly find any papers published with industrial partners (e.g., in the production domain).
However, these would significantly increase the practical relevance of the configurable storages, especially regarding domain-specific requirements.

Interestingly, the configurable systems we examined are mostly not verified or only verified in the context of the product, leading to \textbf{(RO\textsubscript{9})} \emph{a lack of feature-based or family-based versystemsifications}.
We argue that it would be valuable to verify these systems also in these ways, especially in order to fulfill domain-specific requirements and security demands in a verified manner.
Addressing this issue could lead to an increasing confidence in configurable systems and their data storage, including a higher trust in feature interactions.
Moreover, costs could be reduced by avoiding potential system adaptions~\cite{Kenner2021MappingStudy}, e.g., in terms of updates, which could also lead to new security risks.

Surprisingly, about half of the publications are concerned with the evolution of the systems, which is more than in previous studies (e.g.,~\citet{Kenner2021MappingStudy}).
We assume that this fact is maybe due to the papers' focus on cloud systems which are possibly subject of more evolution-related modifications due to their high scalability according to the systems' requirements.
\todo{@Jacob: maybe you have a nice thought regarding evolution..}

Furthermore, we found out that \textbf{(RO\textsubscript{10})} \emph{tools to support the product-line engineering process are usually missing}.
This issues results in the problem that especially publications serving the solution space and thus would be most predestined for practice, lack the assignment of tools allowing the adoption of the proposed configurable systems. 
Thus, it is recommend to at least name the tool or tools used (e.g., FeatureIDE~\cite{Meinicke2017FeatureIDE}) to facilitate the overall comprehensibility and transferability of the systems.

\subsection{Configurable Data Storages and Security}

Not surprisingly, most of the analyzed systems consider security as a non-functional requirement, a quality attribute, or an overall system goal. 
Nevertheless, these considerations also lead to the issue that \textbf{(RO\textsubscript{11})} \emph{security is usually only addressed from a high-level perspective}, leading to a lack of concrete security measures.
However, in four cases security is addressed as one relevant feature of the configurable system and/or its storage.
We argue that considering security as a feature could help to create more awareness and add more attention to security, i.e., relevant threats, risks, and mitigation techniques.
Consequently, security would no longer be one of many goals, but part of the overall system, which is configurable.
Security measures resulting in concrete features (e.g., access control mechanisms based on strong encryption algorithms to avoid brute force attacks) should be modeled nearly equivalent to other features (e.g., in tools such as FeatureIDE~\cite{Meinicke2017FeatureIDE}).

\begin{figure}
	\includegraphics[width=\linewidth]{img/taxonomy.pdf}
	\vspace*{-4ex}
	\caption{Publication overview regarding security goals according to perspectives, numbers indicate amount of publications, middle numbers refer to the overall amount (not a certain sum, since publications can serve multiple perspectives).}
	\label{fig:taxonomy}
	\vspace*{-2ex}
\end{figure}

In addition, we identified that the databases described in the publications are usually not explained in detail, e.g., whether they are SQL or NoSQL databases or how variable they actually are.
We argue that every type or even variant of a database can possibly lead to new requirements which are relevant for systems, models, or patterns interacting with them, e.g., the underlying software system or concrete security measures, e.g. against SQL injection attacks~\cite{humayun2020cyber}.
Moreover, it is not clear what impact the type of databases (e.g., relational) together with certain variability aspects (e.g., version updates) have regarding potential security risks.
Specifically, there only exist reports or recommendations regarding general storage systems without considering variability, e.g., overviews of concrete security challenges of (big) data storage~\cite{CSA2012}.
This is why we argue that \textbf{(RO\textsubscript{12})} \emph{more research is needed in the context of variable storage mediums, i.e., databases, to understand relevant relationships and to minimize security risks cause by storage variability}.

Interestingly, some publications stated variability as one issue that could compromise the security of a configurable system or its storage.
However, although this threat was recognized, only in one case a concrete mitigation technique was proposed, precisely the parallel execution of variants~\cite{nguyen2014exploring}).
Nevertheless, this issue shows that \textbf{(RO\textsubscript{13})} \emph{the treatment of security threats or risks caused by variability and the relationship between variability and the emergence of vulnerabilities need more research}.

Referring to Figure ~\ref{fig:taxonomy}, most of the analyzed configurable systems do not provide actual configurable storages.
Instead, the systems are more concerned with configurable system that also include storage (e.g., to store user data or variability-related data).
Surprisingly, the number of publications in research area of configurable systems that also consider security or privacy is decreasing since 2018.
However, \textbf{(RO\textsubscript{14})} \emph{the research area of configurable systems, especially configurbable storages, and security seems under-explored} since storage is usually interpreted as a part of a system rather than as an individual system with requirements regarding variability.
Consequently, security concepts are often related to the overall system instead of considering the storages themselves.


\subsection{Threats to Validity}

We are aware of some threats that could impair both internal and external validity of our mapping study.
First, the papers' authors usually do not share the same understanding of certain terms and definitions.
This threat is particularly relevant in the context of security goals (e.g., differentiation between authorization and authentication~\cite{ISOIEC270002018}) and what is actually understood as storage, e.g., a database as a storage medium or a cloud system as a storage environment.
Second, there is  a lack of completeness in depth of content and consistency in the context of the analyzed papers.
In detail, some authors described their work in great detail, while others only mentioned actually essential properties (e.g., concrete encryption algorithms) or described them only briefly.
However, this is probably due to the fact that we considered both conference papers and journal articles in our study.
Although we aimed to mitigate potential issues in our data analysis, we cannot ensure that this did not impair the comparability or led to misinterpretations on our side (e.g., in the context of assignments).
Third, we found a few issues regarding the fulfillment of certain criteria, e.g., several publications did not provide any description or even a name of tools supporting their work leading to a decreasing comprehensibility and replicability of their work.
Fourth, besides these threats to the internal validity, the external validity could be impaired by the number of included publications (50) in our study.
Although we already considered three databases (i.e., \textsc{Scopus}, \textsc{IEEE Xplore}, and the \textsc{ACM Guide to Computing Literature}), we are aware of the fact that the smaller the number of included literature, the higher the potential of the impact of misclassifications.
In this context, we assume that we potentially missed papers that are related to the covered topics (i.e., security, configurable systems, data storage) due to our overall search strategy, e.g., the search string that does not cover all potential subtopics such as variability modeling.

Although the described issues may threaten our study findings, we aimed to mitigate them by only relying on established literature databases covering relevant and peer-reviewed publications.
Moreover, we considered a large number of papers in our overall systematical evaluation process (i.e., initially 538 publications fetched based on our search string), leading to a decreased threat of missing publications in this research area.
Thus, we argue that our study provides detailed findings regarding security in the context of configurable systems and story which are valuable for the research community.
For future work, we plan to extend our study by conducting a systematic literature review, e.g. examining the relevant topics and their criteria in more detail based on concrete metrics and proving the identified research trends and gaps.
