\section{Discussion} \label{sec:discussion}

After describing the extraction results, we give an overview of relevant core insights derived from the results.
Precisely, we present 16 \textbf{literature gaps (LG)} we found in the analyzed publications after collaboratively analyzing and discussing the results among this study's authors.
The overall discussion section is oriented towards the three topics of this paper, namely storage, security, and configurable systems. 
Moreover, these three topics are discussed in concert.

\subsection{Storage}
Interestingly, we found out that the concrete understanding of what storage actually is, varies heavily between the selected publications.
In detail, storage is either referred to as concrete medium (e.g., a database), the environment of the medium (e.g., a configurable cloud system), or the cooperation of both, usually with a focus on the environment. 
Our understanding refers to storage as a medium embedded into an environment, which has actually no concrete storing abilities without implementing the medium.
However, we assume that storage is usually used as a generic term for systems which are concerned with certain storage goals or functionalities, leading to \textbf{LG\textsubscript{1}}: \emph{A uniform definition and understanding of storage, especially in the area of configurable systems, is needed} in order to ensure comparable analyses and considerations of storage and the stored data (e.g., in the context of security).

Surprisingly, the uniform assignment of the storage organization structure (i.e., centralized, decentralized, distributed) was hardly possible since the concrete perspective of the publications was not clear.
Precisely, we assume that there exist several possible layers (i.e., perspectives) regarding the organization structure of a system with storage abilities, e.g.,:
\begin{enumerate}[nosep]
	\item the storage medium, e.g., a database,
	\item the storage environment, e.g., a cloud system, 
	\item the software system, e.g. a medical system, and
	\item the storage hardware, e.g., the server system infrastructure.
\end{enumerate}
Each layer might contain systems that are either centralized, decentralized, or distributed.
Since such (\textbf{LG\textsubscript{2}}) \emph{uniform definitions regarding the perspective of the organization structure are not given in any of the papers}, comparable assignments are not or only hardly possible at the moment.
However, we assume that the organizational structure described in the publications usually refers to the storage environment in combination with the software system (e.g., distributed, multi-tenant cloud systems), where the actual storage medium is usually centralized.
Nevertheless, we strongly recommend to do further research in the context of the organizational structure of configurable systems and their storage, e.g. defining the different perspective layers in terms of a framework.

Our results also highlight that usually only the software system or both software system and storage combined are actually variable. 
This finding shows the great dependency and interaction of storage systems and software systems in the context of variability, i.e., both should be considered together if possible.
In addition, the actual configurability of the storage mediums, at least the storage environments which are configurable and scalable by definition (e.g., clouds), is rarely addressed.
This may be due to an insufficient relevance of these systems in the field of product-line engineering or due to a lack of concrete solutions and research in this area.
Nevertheless, we argue that \textbf{(LG\textsubscript{3})} \emph{more research is needed regarding the variability of storage mediums and storage environment, taking into account their interactions with configurable software systems}.

\subsection{Security}
Based on the individual fulfillment of all security-related criteria, we state the \textbf{(LG\textsubscript{4})} \emph{general need for security-related considerations for configurable software systems and their storage} as research in security (including privacy) is lagging behind the technological advances regarding configurable systems and their data storage.
However, this is a already known problem in the context of data storage~\cite{strohbach2016big}.
Furthermore, this need is particularly relevant in the context of addressed norms, standards, or legal regulations. 
Unsurprisingly, usually no security standard is addressed by the publications, although there exist a variety of international standards published by the NIST, ISO, or IEC. 
In only two cases, the most relevant standards of the ISO/IEC 27000 series~\cite{ISOIEC270002018} are described which provide essential definitions (ISO/IEC 27000~\cite{ISOIEC270002018}), requirements (ISO/IEC 27001~\cite{ISOIEC270012013}), or ways of monitoring, measuring, analyzing, and evaluating security concerns (ISO/IEC 27004~\cite{ISOIEC270042016}).
Moreover, relevant legal regulations (usually in the context of privacy for personal (medical) data) are only given in one paper.
Thus, we state that \textbf{(LG\textsubscript{5})} \emph{security for configurable software systems and their storage systems is lacking practice orientation based on established norms, standards, and legal regulations}, leading to a major hurdle in the context of the transfer of theoretical knowledge into business practice.
Possible reasons for not referring to established standards could be the missing need for a practical standards in the context of purely theoretical considerations or the assumption that the configurable applications automatically rely on the ISO/IEC 27000 series as it is one of the most established security standards.
Even the limited distribution of certain systems and variability mechanisms in practice (e.g., configurable robotics cloud applications in Industry 4.0~\cite{gherardi2014software}) may result in an insufficient need for protecting these systems against potential cyber attacks based on established standards.

Our findings also emphasize that \textbf{(LG\textsubscript{6})} \emph{there is a lack of concrete specifications of mitigation techniques} (e.g., concrete encryption algorithms, such as AES-256 or RSA-2048).
For instance, in the context of encryption it is not clear what is encrypted (e.g., data, storage medium, or its communication) and how it is encrypted (e.g., using 1024 or 2048 bit in the context of asymmetric RSA encryption).
We propose the definition of different protection levels and assigned security measures, depending on which data, medium, environment, or system (including variants) need to be protected.
For instance, there is a different need for the protection of personal data such as the birthdate or medical data such as the medical history of a person.
However, in the analyzed papers, security measures are often described in a general way (e.g., firewalls~\cite{syed2016cloud} or access control~\cite{moens2015allocating}) resulting in no or only limited practical relevance in the context of actual cyber attacks (e.g., man-in-the-middle or brute force attacks). 
Especially in the case of configurable systems, i.e., software systems, their storage systems, and the communication between both, a specification that is as precise as possible is essential because of the increased attack surface due to variability~\cite{Kenner2021MappingStudy}.
However, we suggest not only specifying mitigation techniques, but also assigning them to concrete threats or risks (i.e., cyber attacks) to increase the practical relevance.

Interestingly, the perspective of the security technique implementations differ in the analyzed papers similar to data storage.
In detail, some measures refer to the overall software system (e.g., firewall~\cite{siegmund2020dimensions}) while others focus on the storage environment (e.g., access control of cloud users~\cite{moens2015allocating}) or the storage medium (e.g., encryption of a database and its communication~\cite{preuveneers2016systematic}).
However, it is not always clear what exactly a security measure refers to.
Thus, we state that \textbf{(LG\textsubscript{7})} \emph{a concrete assignment of mitigation techniques to their target area (i.e., its perspective) is usually missing}, leading to a decreasing comparability of existing approaches.

Surprisingly, the security goals of the CIA triad (confidentiality, integrity, availability) were not addressed the most (see Figure~\ref{fig:taxonomy}), as stated in related work, e.g.,~\cite{Kenner2021MappingStudy}.
Instead, availability and authorization were usually referred to.
We assume that this is due to the fact that both goals are subjects of typical goals for databases, probably not within the scope of certain security concerns.
Furthermore, we assume that security goals that are relevant but nevertheless have not been mentioned are regarded as given (e.g. integrity and non-repudiation).
In the context of databases, both can be assumed to be automatically fulfilled by the general properties of databases. 
However, we emphasize that \textbf{(LG\textsubscript{8})} \emph{concrete security goals (e.g., according to the definitions given in ISO/IEC 27000~\cite{ISOIEC270002018}) are usually missing in the context of configurable systems}, implying a lack of practical orientation.


\subsection{Configurable Systems}

We identified a variety of diverse domains, implying that most the of selected research is not domain specific. 
Thus, a large number of the analyzed solutions can also be transferred and applied to a large number of cases.
However, we note that this might be a hurdle for actually implementing these approaches in practice, since these solutions are not yet specified for a specific domain or use case. 
Nevertheless, we identified a trend regarding the production domain, more specifically cyber-physical systems and robotics applications.
Consequently, a high application potential for variability in terms of configurable systems that are usually based on cloud environments can be derived for Industry 4.0.

Referring to the projection of the examined systems, most publications seem to focus on the mapping between problem space and solution space or the problem space, meaning that actual (implemented) configurable systems could published more often.
However, if the solution space is not sufficiently addressed, \textbf{(LG\textsubscript{9})} \emph{sufficient concrete approaches that can serve as working examples for practice are missing}. 
Furthermore, we could hardly find any publications published with industry partners (e.g., in the production domain).
However, these would significantly increase the practical relevance of the configurable systems as well as their value for actual implementation, especially regarding domain-specific requirements.
Therfore, we argue that \textbf{(LG\textsubscript{10})} \emph{there is a lack of research in collaboration with practitioners (e.g., industry partners)}.

Interestingly, the configurable systems we examined are mostly not verified or only verified in the context of the product, leading to \textbf{(LG\textsubscript{11})} \emph{a lack of feature-based or family-based verifications}.
We argue that it would be valuable to verify these systems also in these ways, especially in order to fulfill domain-specific requirements and security demands in a verified manner.
Addressing this issue could lead to an increasing confidence in configurable systems and their data storage, including a higher trust in feature interactions.
Moreover, costs could be reduced by avoiding potential system adaptions~\cite{Kenner2021MappingStudy}, e.g., in terms of updates, which could also lead to new security risks.

Surprisingly, about half of the publications are concerned with the evolution of the systems, which is more than in previous studies (e.g.,~\citet{Kenner2021MappingStudy}).
We assume that this fact is maybe due to the papers' focus on cloud systems which are possibly subject of more evolution-related modifications due to their high scalability according to the systems' requirements.
\todo{@Jacob: maybe you have a nice thought regarding evolution..}

Furthermore, we found out that \textbf{(LG\textsubscript{13})} \emph{tools to support the product-line engineering process are usually missing}.
This issues results in the problem that especially publications serving the solution space and thus would be most predestined for practice, lack the assignment of tools allowing the adoption of the proposed configurable systems. 
Thus, it is recommend to at least name the tool or tools used (e.g., FeatureIDE~\cite{Meinicke2017FeatureIDE}) to facilitate the overall comprehensibility and transferability of the systems.

\subsection{Configurable Systems, Data Storage and Security in Concert}

Not surprisingly, most of the analyzed systems consider security as a non-functional requirement, a quality attribute, or an overall system goal. 
Nevertheless, these considerations also lead to the issue that \textbf{(LG\textsubscript{14})} \emph{security is usually only addressed from a high-level perspective}, leading to a lack of concrete security measures.
However, in four cases security is addressed as one relevant feature of the configurable system and/or its storage.
We argue that considering security as a feature could help to create more awareness and add more attention to security, i.e., relevant threats, risks, and mitigation techniques.
Consequently, security would no longer be one of many goals, but part of the overall system, which is configurable.
Security measures resulting in concrete features (e.g., access control mechanisms based on strong encryption algorithms to avoid brute force attacks) should be modeled nearly equivalent to other features (e.g., in tools such as FeatureIDE~\cite{Meinicke2017FeatureIDE}).

\begin{figure}
	\includegraphics[width=\linewidth]{img/taxonomy.pdf}
	\vspace*{-4ex}
	\caption{Publication overview regarding security goals according to perspectives, numbers indicate amount of publications, middle numbers refer to the overall amount (not a certain sum, since publications can serve multiple perspectives).}
	\label{fig:taxonomy}
	\vspace*{-2ex}
\end{figure}

In addition, we identified that the databases described in the publications are usually not explained in detail, e.g., whether they are SQL or NoSQL databases or how variable they actually are.
We argue that every type or even variant of a database can possibly lead to new requirements which are relevant for systems, models, or patterns interacting with them, e.g., the underlying software system or concrete security measures, e.g. against SQL injection attacks~\cite{humayun2020cyber}.
Moreover, it is not clear what impact the general type of databases (e.g., relational), their variability (e.g., in terms of version updates), or the automation of certain database back-end processes (e.g., SQLite for mobile devices) have regarding potential security risks.
This is why we argue that \textbf{(LG\textsubscript{15})} \emph{more research is needed in the context of variable storage mediums, i.e., databases, to understand relevant relationships and to minimize security risks cause by storage variability}.

Interestingly, some publications stated variability as one issue that could compromise the security of a configurable system or its storage.
However, although this threat was recognized, only in one case a concrete mitigation technique was proposed, precisely the parallel execution of variants~\cite{nguyen2014exploring}).
Nevertheless, this issue shows that \textbf{(LG\textsubscript{16})} \emph{the treatment of security threats or risks caused by variability and the relationship between variability and the emergence of vulnerabilities need more research}.

Referring to Figure ~\ref{fig:taxonomy}, most of the analyzed configurable systems do not provide actual configurable storages.
Instead, the systems are more concerned with configurable system that also include storage (e.g., to store user data or variability-related data).
Surprisingly, the number of publications in research area of configurable systems that also consider security or privacy is decreasing since 2018.
However, \textbf{(LG\textsubscript{17})} \emph{the research area of configurable systems, especially configurbable storages, and security seems under-explored} since storage is usually interpreted as a part of a system rather than as an individual system with requirements regarding variability.
Consequently, security concepts are often related to the overall system instead of considering the storages themselves.

\todo{Richard: more examples}


\subsection{Threats to Validity}

We are aware of some threats that could impair both internal and external validity of our mapping study.
First, the papers' authors usually do not share the same understanding of certain terms and definitions.
This threat is particularly relevant in the context of security goals (e.g., differentiation between authorization and authentication~\cite{ISOIEC270002018}) and what is actually understood as storage, e.g., a database as a storage medium or a cloud system as a storage environment.
Second, there is  a lack of completeness in depth of content and consistency in the context of the analyzed papers.
In detail, some authors described their work in great detail, while others only mentioned actually essential properties (e.g., concrete encryption algorithms) or described them only briefly.
However, this is probably due to the fact that we considered both conference papers and journal articles in our study.
Although we aimed to mitigate potential issues in our data analysis, we cannot ensure that this did not impair the comparability or led to misinterpretations on our side (e.g., in the context of assignments).
Third, we found a few issues regarding the fulfillment of certain criteria, e.g., several publications did not provide any description or even a name of tools supporting their work leading to a decreasing comprehensibility and replicability of their work.
Fourth, besides these threats to the internal validity, the external validity could be impaired by the number of included publications (50) in our study.
Although we already considered three databases (i.e., \textsc{Scopus}, \textsc{IEEE Xplore}, and the \textsc{ACM Guide to Computing Literature}), we are aware of the fact that the smaller the number of included literature, the higher the potential of the impact of misclassifications.
In this context, we assume that we potentially missed papers that are related to the covered topics (i.e., security, configurable systems, data storage) due to our overall search strategy, e.g., the search string that does not cover all potential subtopics such as variability modeling.

Although the described issues may threaten our study findings, we aimed to mitigate them by only relying on established literature databases covering relevant and peer-reviewed publications.
Moreover, we considered a large number of papers in our overall systematical evaluation process (i.e., initially 538 publications fetched based on our search string), leading to a decreased threat of missing publications in this research area.
Thus, we argue that our study provides detailed findings regarding security in the context of configurable systems and story which are valuable for the research community.
For future work, we plan to extend our study by conducting a systematic literature review, e.g. examining the relevant topics and their criteria in more detail based on concrete metrics and proving the identified research trends and gaps.
