\section{Discussion} \label{sec:discussion}

\myPar{Storage}


\myPar{Security}
Usually no standard is addressed, In only two cases the most relevant standards of the ISO/IEC 27000 series are described, Furthermore, only in one case legal regulations are addressed leading to the issue that the solutions hardly provide any practical orientation and can hardly be transferred into practice in their described form

security and privacy is lagging behind the technological advances regarding configurable systems and their data storage. Problem in the context of data storage is not new~\cite{strohbach2016big}


\myPar{Configurable Systems}

\myPar{Configurable Systems, Data Storage and Security in Concert}
domains diverse meaning that its usually fundamental research which is not domain specific, more research in collaboration with practitioners is needed
\subsection{Threats to Validity}

We are aware of some threats that could impair both internal and external validity of our mapping study.
First, the papers' authors usually do not share the same understanding of certain terms and definitions.
This threat is particularly relevant in the context of what is actually understood as storage, e.g., a database as a storage medium or a cloud system as a storage environment.
Second, there is  a lack of completeness in depth of content and consistency in the context of the analyzed papers.
In detail, some authors described their work in great detail, while others only mentioned actually essential properties (e.g., concrete encryption algorithms) or described them only briefly.
However, this is probably due to the fact that we considered both conference papers and journal articles in our study.
Although we aimed to mitigate potential issues in our data analysis, we cannot ensure that this did not impair the comparability or led to misinterpretations on our side.
Third, we found a few issues regarding the fulfillment of certain criteria, e.g., several publications did not provide any description or even a name of tools supporting their work leading to a decreasing comprehensibility and replicability of their work.
Fourth, besides these threats to the internal validity, the external validity could be impaired by the number of included publications (50) in our study.
Although we already considered three databases (i.e., \textsc{Scopus}, \textsc{IEEE Xplore}, and the \textsc{ACM Guide to Computing Literature}), we are aware of the fact that the smaller the number of included literature, the higher the potential of the impact of misclassifications.
In this context, we assume that we potentially missed papers that are related to the covered topics (i.e., security, configurable systems, data storage) due to our overall search strategy, e.g., the search string that does not cover all potential subtopics such as variability modeling.

Although the described issues may threaten our study findings, we aimed to mitigate them by only relying on established literature databases covering relevant and peer-reviewed publications.
Moreover, we considered a large number of papers in our overall systematical evaluation process (i.e., initially 538 publications fetched based on our search string), leading to a decreased threat of missing publications in this research area.
Thus, we argue that our study provides detailed findings regarding security in the context of configurable systems and story which are valuable for the research community.
For future work, we plan to extend our study by conducting a systematic literature review, e.g. examining the relevant topics and their criteria in more detail based on concrete metrics and proving the identified research trends and gaps. 
